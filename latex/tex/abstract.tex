\section*{Abstract}

The analytical cross section for the parton-level \(\qqgg\) diphoton
process is calculated.  With the application to this in mind, some
monte carlo integration and sampling methods are studied, implemented
and compared. Proton-Proton scattering is simulated on the partonic
level through the use of parton density functions. Throughout,
obtained results are verified through comparison with the \sherpa\
event generator. Finally, a more holistic phenomenological study is
performed on the diphoton process by simulating additional effects
including parton showers, hadronization and multiple interactions at
\lhc conditions with the \sherpa\ event generator.

\section*{Zusammenfassung}

Der analytische wirkungsquerschnitt fuer den \(\qqgg\) diphoton prozess
wird berechnet. Mit hinblick auf die Anwendung auf diesen Prozess
werden monte carlo Methoden untersucht, implementiert und
verglichen. Proton-Proton Streuung wird auf der partonischen ebene
mithilfe von parton-density Funktionen simuliert.  Gewonnene resultate
werden durchweg mit dem \sherpa\ event generator verifiziert. Zuletzt
wird der diphoton Prozess mit dem \sherpa\ event geneeiner
realistischere Simulation, die unter anderem parton shower,
hadronisierung und multiple interactions ber\"ucksichtigt.
