\section*{Abstract}

The analytical cross section for the parton-level \(\qqgg\) diphoton
process is calculated in leading order. With the application to this
process in mind, some Monte Carlo methods for integration and sampling
are studied, implemented and compared. The hard diphoton process in
Proton-Proton scattering is simulated in leading order using parton
density functions. Throughout, the obtained results are verified
through comparison with the \sherpa\ event generator. Finally, a more
holistic phenomenological study is performed on the diphoton process
by simulating and comparing additional effects including parton
showers, hadronization and multiple interactions at \lhc\ conditions
with the \sherpa\ event generator.


\section*{Zusammenfassung}

Der analytische Wirkungsquerschnitt fuer den \(\qqgg\) diphoton
Prozess wird in f\"uhrender Ordnung berechnet. In Hinblick auf die
Anwendung auf diesen Prozess werden Monte Carlo Methoden f\"ur
Integration und zur Generierung von Stichproben aus
Warscheinlichkeitsverteilungen untersucht, implementiert und
verglichen. Der harte diphoton Prozess wird in der Proton-Proton
Streuung mithilfe von Partondichtefunktionen in f\"uhrender Ordnung
simuliert. Gewonnene Resultate werden durchweg mit dem \sherpa\ Event
Generator verifiziert. Zuletzt wird der diphoton Prozess mit dem
\sherpa\ Event Generator in einer realistischeren Simulation
untersucht, die unter anderem Parton Shower, Hadronisierung und
Multiple Interactions ber\"ucksichtigt und deren Einfluesse
vergleicht.

%%% Local Variables: ***
%%% mode: latex ***
%%% TeX-master: "../document.tex"  ***
%%% End: ***
