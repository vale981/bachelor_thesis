\chapter{Appendix}%
\label{chap:appendix}

\section{Sherpa Runcards}%
\label{sec:runcards}

\subsection{Quark Antiquark Anihilation}%
\label{sec:qqggruncard}
\yamlfile{../prog/runcards/qqgg/Sherpa.yaml}

\subsection{Proton Proton Scattering}%
\label{sec:ppruncard}
\yamlfile{../prog/runcards/pp/Sherpa.yaml}

\subsection{Holistic Proton Proton Scattering}%
\label{sec:ppruncardfull}
\VerbatimInput{../prog/runcards/pp_phaeno_299_port/runcards/with_pT_and_fragmentation_and_mi/Run.dat}


\section{Python Code}%
\label{sec:pycode}
The code listed here is a proof of concept and by no means optimized
for performance or even elegance.

\subsection{MC Methods}%
\label{sec:mcpy}
The python source code can be found at
\href{https://github.com/vale981/bachelor_thesis/blob/master/prog/python/qqgg/monte_carlo.py}{github}.
% \pythonfile{../prog/python/qqgg/monte_carlo.py}

\section{Rivet Analysis Code}%
\label{sec:rivetcode}

\subsection{Simple Diphoton Analysis}%
\label{sec:simpdiphotriv}
\cppfile{../prog/analysis/qqgg_simple/MC_DIPHOTON_SIMPLE.cc}

\subsection{Proton Proton Scattering Analysis}%
\label{sec:ppanalysis}
\cppfile{../prog/runcards/pp/qqgg_proton/MC_DIPHOTON_PROTON.cc}

\subsection{Holistic Proton Scattering Analysis}%
\label{sec:ppanalysisfull}
\cppfile{../prog/runcards/pp_phaeno_299_port/qqgg_proton/MC_DIPHOTON_PROTON.cc}

\section{Mathematical Notes}%
\label{sec:matap}

\subsection{Equivalence of Importance Sampling and Change of
  Variables}%
\label{sec:equap}

Assume the same prerequisites as in \cref{sec:mcint}. Here the proof
is made for two dimensions, higher dimension follow analogous (but
with a burden of notation). In truth, a multidimensional integral can
always be reduced to a series of one dimensional integrals, but doing
the calculation for the two dimensional case is illustrative.

Define a variable transformation as in \cref{eq:rfuncsap}, where the
inverses are taken with reference to the variables before the
semicolon and \(a,b\in [0, 1]\). The inverse can be taken, as
\(\rho > 0\).

\begin{equation}
  \label{eq:rfuncsap}
  \begin{split}
  R_x(x) &= \int_y\int_0^x\rho(x, y) \dd{x}\dd{y} \\
  R_y(y; x) &= \frac{\int_0^y\rho(x, y) \dd{x}\dd{y}}{\int_y
              f(x,y)\dd{y}} \\
  \vb{x} &= \mqty(x \\ y) = \mqty(R_x^{-1}(a) \\ R_y^{-1}(b, x(a)))
  \end{split}
\end{equation}

The Jacobian determinant is thus given by \cref{eq:jacap}.
\begin{equation}
  \label{eq:jacap}
  (\partial_a R_x^{-1}(a))\cdot (\partial_b R_y^{-1}(b; x(a))) =
  \frac{1}{\int_y\rho(x(a), y)\dd{y}}\cdot \frac{\int_y\rho(x(a),
    y)\dd{y}}{\rho(x(a), y(a, b))} = \frac{1}{\rho(x(a), y(a, b))}
\end{equation}

The integral \cref{eq:baseintegral} becomes \cref{eq:newintap} which
is the same as if \(f\) (interpreted as a probability density) was
transformed to the variables \(a, b\).

\begin{equation}
  \label{eq:newintap}
  \int_\Omega
  \qty[\frac{f(\vb{x})}{\rho(\vb{x})}] \rho(\vb{x}) \dd{\vb{x}} =
  \int_0^1\int_0^1 \frac{f(x(a), y(a,b))}{\rho(x(a), y(a, b))} \dd{a}\dd{b}
\end{equation}

That means taking sampling points \(\{\vb{x_i}\}\sim \rho\) and
weighting samples with \(1/\rho(\vb{x})\) is equivalent to taking
uniformly distributed samples of \(f\) in \(a,b\) space with the
appropriate transformation.

This works, because the transformation law \cref{eq:rfuncsap}
uniformly distributed samples into samples distributed like \(\rho\).

%%% Local Variables:
%%% mode: latex
%%% TeX-master: "../document"
%%% End:
