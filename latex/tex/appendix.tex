\chapter{Appendix}%
\label{chap:appendix}

\section{Sherpa Runcards}%
\label{sec:runcards}

\subsection{Quark Antiquark Anihilation}%
\label{sec:qqggruncard}
\yamlfile{../prog/runcards/qqgg/Sherpa.yaml}

\subsection{Proton Proton Scattering}%
\label{sec:ppruncard}
\yamlfile{../prog/runcards/pp/Sherpa.yaml}

\subsection{Holistic Proton Proton Scattering}%
\label{sec:ppruncardfull}
\VerbatimInput{../prog/runcards/pp_phaeno_299_port/runcards/with_pT_and_fragmentation_and_mi/Run.dat}


\section{Python Code}%
\label{sec:pycode}
The code listed here is a proof of concept and by no means optimized
for performance or even elegance.

\subsection{Monte Carlo Methods}%
\label{sec:mcpy}
The python source code can be found at
\href{https://github.com/vale981/bachelor_thesis/blob/master/prog/python/qqgg/monte_carlo.py}{github}.
% \pythonfile{../prog/python/qqgg/monte_carlo.py}

\section{Rivet Analysis Code}%
\label{sec:rivetcode}

\subsection{Simple Diphoton Analysis}%
\label{sec:simpdiphotriv}
\cppfile{../prog/analysis/qqgg_simple/MC_DIPHOTON_SIMPLE.cc}

\subsection{Proton Proton Scattering Analysis}%
\label{sec:ppanalysis}
\cppfile{../prog/runcards/pp/qqgg_proton/MC_DIPHOTON_PROTON.cc}

\subsection{Holistic Proton Scattering Analysis}%
\label{sec:ppanalysisfull}
\cppfile{../prog/runcards/pp_phaeno_299_port/qqgg_proton/MC_DIPHOTON_PROTON.cc}

%%% Local Variables:
%%% mode: latex
%%% TeX-master: "../document"
%%% End:
