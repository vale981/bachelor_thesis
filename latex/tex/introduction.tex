\chapter{Introduction}%
\label{chap:intro}

Monte Carlo (MC) methods have been and still are one of the most important
tools for numerical calculations in particle physics. Be it for
validating the well established standard model or for making
predictions about new theories, Monte Carlo simulations are the
crucial interface of theory and experimental data, making them
directly comparable. Furthermore horizontal scaling is almost trivial
to implement in Monte Carlo algorithms, making them well adapted to
modern parallel computing. In this thesis, the use of Monte Carlo
methods will be traced through from simple integration to the
simulation of proton-proton scattering.

The ``Drosophila'' of this thesis is the quark annihilation into two
photons \(\qqgg\), henceforth called the diphoton process. It forms an
important background to the higgs decay channel
\(H\rightarrow \gamma\gamma\) and to a dihiggs decay
\(HH\rightarrow b\bar{b}\gamma\gamma\)~\cite{aaboud2018:sf}, while
still being a pure QED process and thus calculable by hand within the
scope of this thesis. The differential and total cross section of this
process is being calculated in leading order in \cref{chap:qqgg} and
the obtained result is compared to the total cross section obtained
with the \sherpa~\cite{Gleisberg:2008ta} event generator, used as
matrix element integrator. In \cref{chap:mc} some simple Monte Carlo
methods are discussed, implemented and their results
compared. Beginning with a study of Monte Carlo integration the
\vegas\ algorithm~\cite{Lepage:19781an} is implemented and
evaluated. Subsequently Monte Carlo sampling methods are explored and
the output of \vegas\ is used to improve the sampling
efficiency. Histograms of observables are generated and compared to
histograms from \sherpa\ using the \rivet~\cite{Bierlich:2019rhm}
analysis framework. \Cref{chap:pdf} deals with proton-proton
scattering in the partonic picture using parton density functions,
ending with the implementation of a simple event generator for
\(\ppgg\) scattering at \lhc\ conditions. Some integration and
sampling algorithms and their implementation are adapted to the
multidimensional case and histograms of observables are generated with
good efficiency.  Because a real \(pp\) scattering event also
incorporates processes like parton showers, hadronization and multiple
interactions, a realistic simulation should account for those
effects. The impact of those effects on observables is studied in
\cref{chap:pheno} using the \sherpa\ event generator.

\section{Conventions}%
\label{sec:convent}

Throughout natural units with \(c=1, \hbar = 1, k_B=1, \varepsilon_0
= 1\) are used unless stated otherwise. Histograms without label on
the y-axis are normalized to unity and to be interpreted as
probability densities.

\section{Source Code}%
\label{sec:source}

The (literate) python code, used to generate most of the results and
figures can be found on under
\url{https://github.com/vale981/bachelor_thesis/} and more
specifically in the subdirectory \texttt{prog/python/qqgg}.

The file \texttt{monte\_carlo.py} implements all the monte-carlo
algorithm related functionality as a module. The file
\texttt{analytical\_xs.org} contains a literate computation notebook
that generates all the results of \cref{chap:mc}. The file
\texttt{parton\_density\_function\_stuff.org} contains all the
computations for \cref{chap:pdf}. The python code makes heavy use of
\href{https://www.scipy.org/}{scipy} (and of course
\href{https://numpy.org/}{numpy}).

%%% Local Variables: ***
%%% mode: latex ***
%%% TeX-master: "../document.tex"  ***
%%% End: ***
