\chapter{Introduction}%
\label{chap:intro}

Monte Carlo (MC) methods have been and still are one of the most
important tools for numerical calculations in particle physics. Be it
for validating the well established Standard Model (SM) or for making
predictions about new theories, MC simulations are the crucial
interface of theory and experimental data, making them directly
comparable.
% Furthermore horizontal scaling is almost trivial to implement in MC
% algorithms, making them well adapted to modern parallel computing.
In this thesis, the use of MC methods will be traced through from
simple integration to the simulation of proton-proton scattering.

The test subject here is the quark-antiquark annihilation into two
photons \(\qqgg\), henceforth called the diphoton process. It forms an
important background to the Higgs decay channel
\(H\rightarrow \gamma\gamma\), which was instrumental in its
discovery~\cite{Aad:2012tfa,Chatrchyan:2012ufa}, and to a dihiggs
decay \(HH\rightarrow b\bar{b}\gamma\gamma\), a process of recent
interest~\cite{aaboud2018:sf} to study the higgs self coupling and
probe the limits of the SM. All the while, the diphoton process is
still pure QED (Quantum Electrodynamics) at leading order and thus
calculable by hand within the scope of this thesis as is being done in
\cref{chap:qqgg}. The obtained result is compared to the total cross
section obtained with the \sherpa~\cite{Gleisberg:2008ta} event
generator, used as matrix element integrator. In \cref{chap:mc} some
simple MC methods are discussed, implemented and their results
compared. After studying some basic MC integration methods, the
\vegas\ algorithm~\cite{Lepage:19781an} is implemented and
evaluated. Subsequently MC sampling methods, which are closely related
to the integration methods, are explored and the output of \vegas\ is
used to improve the sampling efficiency. Histograms of observables are
generated and compared to histograms from \sherpa\ using the
\rivet~\cite{Bierlich:2019rhm} analysis framework. \Cref{chap:pdf}
deals with proton-proton scattering in the partonic picture using
parton density functions, ending with the implementation of a simple
event generator for \(\ppgg\) scattering at \lhc\ conditions. Some
integration and sampling algorithms and their implementations are
adapted to the multidimensional case and histograms of observables are
generated with good efficiency. Because a real \(pp\) scattering event
also entails processes like parton showers, hadronization and multiple
interactions, a realistic simulation must account for those
effects. The impact of those effects on observables is studied in
\cref{chap:pheno} using the \sherpa\ event generator.

\section{Conventions}%
\label{sec:convent}

Throughout, natural units with
\(c=1, \hbar = 1, k_B=1, \varepsilon_0 = 1\) are used unless stated
otherwise. The fine structure constant's value \(\alpha = 1/137.036\)
is configured in \sherpa\ and used in analytic calculations.

\section{Source Code}%
\label{sec:source}

The (literate) \texttt{Python} code, used to generate most of the
results and figures can be found under
\url{https://github.com/vale981/bachelor_thesis/} and more
specifically in the subdirectory \texttt{prog/python/qqgg}.

The file \texttt{monte\_carlo.py} implements all the monte-carlo
algorithm related functionality as a module. The file
\texttt{analytical\_xs.org} contains a literate computation notebook
that generates all the results of \cref{chap:mc}. The file
\texttt{parton\_density\_function\_stuff.org} contains all the
computations for \cref{chap:pdf}. The \texttt{Python} code makes heavy
use of \href{https://www.scipy.org/}{scipy}~\cite{2020Virtanen:Sc} and
its component \href{https://numpy.org/}{numpy}).

\section{Software Versions}%
\label{sec:versions}

In \cref{sec:compsher,chap:mc} the development version of \sherpa\ has
been used. \Cref{chap:pdf,chap:pheno} use version \texttt{2.2.10} for
reasons of stability.

In the whole thesis, the version \texttt{3.1.0} of \rivet\ was used.

%%% Local Variables: ***
%%% mode: latex ***
%%% TeX-master: "../document.tex"  ***
%%% End: ***
