\chapter{Introduction}%
\label{chap:intro}

Monte carlo methods have been and still are one of the most important
tools for theoretical calculations in particle physics. Be it for
validating the well established standard model or for making
predictions about new theories, monte carlo simulations are the
crucial \emph{adapter} of theory and experimental data, making them
directly comparable. Furthermore horizontal scaling is almost trivial
to implement in monte carlo algorithms, making them well adapted to
modern parallel computing.

The \emph{Drosophila} of this thesis is the quark annihilation into
two photons \(\qqgg\), henceforth called the diphoton process. It
forms an important background to the higgs decay channel
\(H\rightarrow \gamma\gamma\) and a process of recent interest
\(HH\rightarrow b\bar{b}\gamma\gamma\)~\cite{aaboud2018:sf}, while
still being a pure QED process. The differential and total cross
section of this process is being calculated in leading order in
\cref{chap:qqgg} and the obtained result is compared to the total
cross section obtained with the \sherpa~\cite{Gleisberg:2008ta} event
generator, used as matrix element integrator. In \cref{chap:mc} some
simple monte carlo methods are discussed, implemented and their
results compared. First monte carlo integration is studies and the
\vegas\ algorithm~\cite{Lepage:19781an} is implemented and
evaluated. Subsequently monte carlo sampling methods are explored and
the output of \vegas\ is used to improve the sampling
efficiency. Histograms of observables are generated and compared to
histograms from \sherpa using the \rivet~\cite{Bierlich:2019rhm}
analysis framework. \Cref{chap:pdf} deals with proton proton
scattering in the partonic picture using parton density functions,
resulting in the implementation of a simple event generator for
\(\ppgg\) scattering. The integration and sampling algorithms and
their implementation are adapted to the mulidimensional case and
histograms of observables are generated with good efficiency.  Because
a real \(pp\) scattering event also incorporates processes like parton
showers, hadronization and multiple interactions, a realistic
simulation should account for those effects. The impact of those
effects on observables is studied in \cref{chap:pheno} using the
\sherpa event generator.

%%% Local Variables: ***
%%% mode: latex ***
%%% TeX-master: "../document.tex"  ***
%%% End: ***
