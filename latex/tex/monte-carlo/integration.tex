\section{Monte Carlo Integration}%
\label{sec:mcint}

Consider a function
\(f\colon \vb{x}\in\Omega\subset\mathbb{R}^n\mapsto\mathbb{R}\) and a
probability density on
\(\rho\colon \vb{x}\in\Omega\mapsto\mathbb{R}_{\geq 0}\) with
\(\int_{\Omega}\rho(\vb{x})\dd{\vb{x}} = 1\).  By multiplying \(f\)
with a one in the fashion of \cref{eq:baseintegral}, the integral of
\(f\) over \(\Omega\) can be interpreted as the expected value
\(\EX{F/\Rho}\) of the random variable \(F/\Rho\) under the
distribution \(\rho\). This is the key to most MC methods.
%
\begin{equation}
  \label{eq:baseintegral}
  I = \int_\Omega f(\vb{x}) \dd{\vb{x}} = \int_\Omega
  \qty[\frac{f(\vb{x})}{\rho(\vb{x})}] \rho(\vb{x}) \dd{\vb{x}} = \EX{\frac{F}{\Rho}}
\end{equation}
%
The expected value \(\EX{F/\Rho}\) can be approximated by calculating
the mean of \(F/\Rho\) with \(N\) finite samples
\(\{\vb{x}_i\}_{i\in\overline{1,N}}\sim\rho\) (distributed according
to \(\rho\)), where \(N\) is a very large integer.
%
\begin{equation}
  \label{eq:approxexp}
  \EX{\frac{F}{\Rho}} \approx
  \frac{1}{N}\sum_{i=1}^N\frac{f(\vb{x_i})}{\rho(\vb{x_i})}
  \xrightarrow{N\rightarrow\infty} I
\end{equation}
%
The convergence of \cref{eq:approxexp} is due to the nature of the
expected value \cref{eq:evalue-mean} and
variance \cref{eq:variance-mean} of the mean
\(\overline{X} = \frac{1}{N}\sum_i X_i\) of \(N\) uncorrelated random
variables \(\{X_i\}_{i\in\overline{1,N}}\) with the same distribution,
expected value \(\EX{X_i}=\mathbb{E}\) and variance
\(\sigma_i^2 = \sigma^2\).
%
\begin{gather}
  \EX{\overline{X}} = \frac{1}{N}\sum_i\EX{X_i} = \mathbb{E} \label{eq:evalue-mean}\\
  \sigma^2_{\overline{X}} = \sum_i\frac{\sigma_i^2}{N^2} =
                            \frac{\sigma^2}{N}  \label{eq:variance-mean}
\end{gather}
%
Because \(\frac{\sigma^2}{N}\xrightarrow{N\rightarrow\infty} 0\)
\cref{eq:approxexp} really converges to \(I\). For finite \(N\) the
value of \cref{eq:approxexp} varies around \(I\) with the variance
\(\VAR{F/\Rho}\cdot N^{-1}\) as in \cref{eq:varI}.
%
\begin{align}
  \VAR{\frac{F}{\Rho}} &= \int_\Omega \qty[I -
  \frac{f(\vb{x})}{\rho(\vb{x})}]^2 \rho({\vb{x}}) \dd{\vb{x}} =
  \int_\Omega \qty[\qty(\frac{f(\vb{x})}{\rho(\vb{x})})^2 -
  I^2]\rho({\vb{x}}) \dd{\vb{x}}   \label{eq:varI}
 \\
  &\approx \frac{1}{N - 1}\sum_i \qty[I -
  \frac{f(\vb{x_i})}{\rho(\vb{x_i})}]^2  \label{eq:varI-approx}
\end{align}
%
The goal now is to reduce \(\VAR{F/\Rho}\) to speed up the convergence
of \cref{eq:approxexp} and achieve higher accuracy with fewer function
evaluations. There are at least three angles of attack
in~\ref{eq:baseintegral}, namely the distribution \(\rho\), the
variable \(\vb{x}\), and the integration volume
\(\Omega\). Accordingly some ways variance reductions can be
accomplished are choosing a suitable \(\rho\) (importance sampling),
by transforming the integral onto another variable or by subdividing
integration volume into several sub-volumes of different size while
keeping the sample size constant in all sub-volumes (stratified
sampling).\footnote{There are of course still other methods like the
  multi-channel method.}\footnote{Of course, combinations of these
  methods can be applied as well.}  Combining ideas from importance
sampling and stratified sampling leads to the \vegas\
algorithm~\cite{Lepage:19781an} that approximates the optimal
distribution of importance sampling by adaptive subdivision of the
integration volume into a grid.

The convergence of \cref{eq:approxexp} is not dependent on the
dimensionality of the integration volume as opposed to many other
numerical integration algorithms (trapezoid rule, Simpsons rule) that
usually converge like \(N^{-\frac{k}{n}}\) with
\(k\in\mathbb{N}_{>0}\) and \(n\) being the dimensionality as opposed
to \(N^{-\frac{k}{n}}\) with MC. Because phase space integrals in
particle physics usually have a high dimensionality, MC integration is
a suitable approach there. When implementing MC methods, the random
samples can be obtained through hardware or software random number
generators (RNGs). Most implementations utilize software RNGs because
they supply pseudo-random numbers in a reproducible way, which
facilitates deniability and comparability~\cite{buckley:2011ge}.
% TODO: maybe remove, ask Frank

\subsection{Naive Monte Carlo Integration and Change of Variables}
\label{sec:naivechange}

The simplest choice for \(\rho\) is given
by \cref{eq:simplep}, the uniform distribution.
%
\begin{equation}
  \label{eq:simplep}
  \rho(\vb{x}) = \frac{1}{\int_{\Omega}1\dd{\vb{x'}}} =
  \frac{1}{\abs{\Omega}}
\end{equation}
%
With this distribution \cref{eq:approxexp}
becomes \cref{eq:approxexp-uniform}. In other words, \(I\) is just
the mean of \(f\) in \(\Omega\), henceforth
called \(\bar{f}\), multiplied with the volume.
%
\begin{equation}
  \label{eq:approxexp-uniform}
  \EX{\frac{F}{\Rho}} \approx
  \frac{\abs{\Omega}}{N}\sum_{i=1}^N f(\vb{x_i}) = \abs{\Omega}\cdot\bar{f}
\end{equation}
%
The variance \(\VAR{I}=\VAR{F/\Rho}\) is now given
by \cref{eq:approxvar-I}. Note that the factor \(\abs{\Omega}\) gets
squared when approximating the integral by the sum.
%
\begin{equation}
  \label{eq:approxvar-I}
  \VAR{I} = \abs{\Omega}\int_\Omega f(\vb{x})^2 -
  \underbrace{\qty(\frac{I}{\abs{\Omega}})^2}_{=\bar{f}} \dd{\vb{x}} \equiv \abs{\Omega}\cdot\sigma_f^2 \approx
  \frac{\abs{\Omega}^2}{N-1}\sum_{i}\qty[f(\vb{x}_i) - \bar{f}]^2
\end{equation}
%
Applying this method to integrate the \(\qqgg\) cross section from
\cref{eq:crossec} over a \(\theta\) interval, equivalent to
\(\eta\in [-2.5, 2.5]\) with a target accuracy of
\(\varepsilon=10^{-3}\) results in \result{xs/python/xs_mc} with a
sample size of \result{xs/python/xs_mc_N}.

Changing variables and integrating \cref{eq:xs-eta} over \(\eta\) with
the same target accuracy yields~\result{xs/python/xs_mc_eta} with a
sample size of just~\result{xs/python/xs_mc_eta_N}. The dramatic
reduction in variance and sample size can be understood qualitatively
by studying \cref{fig:xs-int-comp}, which shows both integrands with
the same y-axis scaling and their standard deviation visualized. The
differential cross section in terms of
\(\eta\)~(\cref{fig:xs-int-eta}) is less steep than the differential
cross section in terms of \(\theta\)~(\cref{fig:xs-int-theta}) and
takes on large values over most of the integration interval. In
general, the Jacobian arising in variable transformation has the same
effect as the probability density in importance sampling. It can be
shown that importance sampling and change of variables are formally
equivalent (see \ref{sec:equap}).
%
\begin{figure}[ht]
  \centering
  \begin{subfigure}[c]{.49\textwidth}
    \plot{xs/xs_integrand}
    \caption[\(2\pi\dv{\sigma}{\theta}\) with integration
    boundaries]{\label{fig:xs-int-theta} The integrand arising from
      differential cross section \(\dv{\sigma}{\theta}\) with the
      integration borders visualized as gray lines.}
  \end{subfigure}
  \begin{subfigure}[c]{.49\textwidth}
    \plot{xs/xs_integrand_eta}
    \caption[Differential cross section for \(\qqgg\) with integration
    boundaries]{\label{fig:xs-int-eta} The differential cross section
      \(\dv{\sigma}{\eta}\) (see \cref{eq:xs-eta}) scaled by \(2\pi\)
      with the integration borders visualized as gray lines.}
  \end{subfigure}
  \caption{\label{fig:xs-int-comp} Comparison of two parametrisations
    of the differential cross section. The same y-axis scaling has
    been chosen to visualize the difference in variance.}
\end{figure}
%
\subsection{Integration with \vegas}
\label{sec:mcintvegas}

Stratified sampling gives optimal results, when the variance in every
sub-volume is the same~\cite{Lepage:19781an}. In importance sampling,
the optimal probability distribution is given
by \cref{eq:optimalrho}, where \(f(\Omega) \geq 0\) is presumed
without loss of generality. When applying \vegas\ to multi dimensional
integrals,~\cref{eq:optimalrho} is usually modified to factorize into
distributions for each variable to simplify calculations.
%
\begin{equation}
  \label{eq:optimalrho}
  \rho(\vb{x}) = \frac{f(\vb{x})}{\int_\Omega f(\vb{y})\dd{y}}
\end{equation}
%
The idea behind \vegas\ is to subdivide \(\Omega\) into hypercubes
(create a grid), define \(\rho\) as step-function with constant value
on those hypercubes and iteratively approximating
\cref{eq:optimalrho}, instead of trying to minimize the variance
directly. In the end, the samples are concentrated where \(f\) takes
on the highest values and changes most rapidly. This is done by
subdividing the hypercubes into smaller chunks, based on their
contribution to the integral, which is calculated through MC sampling,
and then varying the hypercube borders until all hypercubes contain
the same number of chunks.  So if the contribution of a hypercube is
large, it will be divided into more chunks than others. When the
hypercube borders are then shifted, this hypercube will shrink, while
others will grow by consuming the remaining chunks.  Repeating this
step in so called \vegas\ iterations will converge the contribution of
each hypercube to the same value. More details about the algorithm can
be found in~\cite{Lepage:19781an}.  Note that no knowledge about the
integrand is required. The probability density used by \vegas\ is
given in \cref{eq:vegasrho} with \(K\) being the number of hypercubes
and \(\Omega_i\) being the hypercubes themselves.
%
\begin{equation}
  \label{eq:vegasrho}
  \rho(\vb{x}) = K\cdot
  \begin{cases}
    \frac{1}{\abs{\Omega_i}} & \vb{x}\in\Omega_{i\in\overline{1,K}} \\
    0 & \text{otherwise}
  \end{cases}
\end{equation}
%
\begin{figure}[ht]
  \centering \plot{xs/xs_integrand_vegas}
  \caption[\(2\pi\dv{\sigma}{\theta}\) scaled to increments found by
  \vegas\ ]{\label{fig:xs-int-vegas} The same integrand as in
    \cref{fig:xs-int-theta} with \vegas-generated increments and
    weighting applied (\(f/\rho\)). The colored bands are the standard
    deviations of the distributions with matching color.}
\end{figure}
%
This algorithm has been implemented in python and applied to
\cref{eq:crossec}.  In one dimension the hypercubes become simple
interval increments and applying \vegas\ to \cref{eq:crossec} with
\result{xs/python/xs_mc_θ_vegas_K} increments yields
\result{xs/python/xs_mc_θ_vegas} with
\result{xs/python/xs_mc_θ_vegas_N} function evaluations (including
\vegas\ iterations). This result is comparable with tho one obtained
by parameter transformation in \cref{sec:naivechange}.  The sample
count \(N\) is the total number of evaluations of \(f\). The resulting
increments and the weighted integrand \(f/\rho\) are depicted in
\cref{fig:xs-int-vegas}, along with the original integrand and it is
intuitively clear, that the variance is being reduced. Smaller
increments correspond to higher sample density and lower weights,
flattening out the integrand.

Generally the result gets better with more increments, but at the cost
of more \vegas\ iterations. The intermediate values of those
iterations can be accumulated to improve the accuracy of the end
result~\cite[197]{Lepage:19781an}.

The \vegas\ algorithm can be adapted to \(n\) dimensions by using a
grid of hypercubes instead of intervals and using the algorithm along
each axis with a slightly altered weighting
mechanism~\cite[197]{Lepage:19781an}.

%%% Local Variables:
%%% mode: latex
%%% TeX-master: "../../document"
%%% End:
