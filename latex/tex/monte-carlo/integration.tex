%%% Local Variables: ***
%%% mode: latex ***
%%% TeX-master: "../../document.tex"  ***
%%% End: ***

\section{Monte-Carlo Integration}%
\label{sec:mcint}

Consider a function
\(f\colon \vb{x}\in\Omega\subset\mathbb{R}^n\mapsto\mathbb{R}\) and a
probability density on
\(\rho\colon \vb{x}\in\Omega\mapsto\mathbb{R}_{\geq 0}\) with
\(\int_{\Omega}\rho(\vb{x})\dd{\vb{x}} = 1\).  By multiplying \(f\)
with a \(1\) in the fashion of~\eqref{eq:baseintegral}, the Integral
of \(f\) over \(\Omega\) can be interpreted as the expected value
\(\EX{F/\Rho}\) of the random variable \(F/\Rho\) under the
distribution \(\rho\). This is the key to most monte-carlo methods.

\begin{equation}
  \label{eq:baseintegral}
  I = \int_\Omega f(\vb{x}) \dd{\vb{x}} = \int_\Omega
  \qty[\frac{f(\vb{x})}{\rho(\vb{x})}] \rho(\vb{x}) \dd{\vb{x}} = \EX{\frac{F}{\Rho}}
\end{equation}

The expected value \(\EX{F/\Rho}\) can be approximated by taking the
mean of \(F/\Rho\) with \(N\) finite samples
\(\{\vb{x}_i\}_{i\in\overline{1,N}}\sim\rho\) (distributed according
to \(\rho\)), where \(N\) is a very large integer.

\begin{equation}
  \label{eq:approxexp}
  \EX{\frac{F}{\Rho}} \approx
  \frac{1}{N}\sum_{i=1}^N\frac{f(\vb{x_i})}{\rho(\vb{x_i})}
  \xrightarrow{N\rightarrow\infty} I
\end{equation}

The convergence of~\eqref{eq:approxexp} is due to the nature of the
expected value~\eqref{eq:evalue-mean} and
variance~\eqref{eq:variance-mean} of the mean
\(\overline{X} = \frac{1}{N}\sum_i X_i\) of \(N\) uncorrelated random
variables \(\{X_i\}_{i\in\overline{1,N}}\) with the same distribution,
expected value \(\EX{X_i}=\mathbb{E}\) and variance
\(\sigma_i^2 = \sigma^2\).

\begin{gather}
  \EX{\overline{X}} = \frac{1}{N}\sum_i\EX{X_i} = \mathbb{E} \label{eq:evalue-mean}\\
  \sigma^2_{\overline{X}} = \sum_i\frac{\sigma_i^2}{N^2} =
                            \frac{\sigma^2}{N}  \label{eq:variance-mean}
\end{gather}

Because
\(\frac{\sigma^2}{N}\xrightarrow{N\rightarrow\infty}
0\)~\eqref{eq:approxexp} really converges to \(I\). For finite \(N\)
value of~\eqref{eq:approxexp} varies around \(I\) with the variance

\(\VAR{F/\Rho}\cdot N^{-1}\) as in~\eqref{eq:varI}.

\begin{align}
  \VAR{\frac{F}{\Rho}} &= \int_\Omega \qty[I -
  \frac{f(\vb{x})}{\rho(\vb{x})}]^2 \rho({\vb{x}}) \dd{\vb{x}} =
  \int_\Omega \qty[\qty(\frac{f(\vb{x})}{\rho(\vb{x})})^2 -
  I^2]\rho({\vb{x}}) \dd{\vb{x}}   \label{eq:varI}
 \\
  &\approx \frac{1}{N - 1}\sum_i \qty[I -
  \frac{f(\vb{x_i})}{\rho(\vb{x_i})}]^2  \label{eq:varI-approx}
\end{align}

The name of the game now is to reduce \(\VAR{F/\Rho}\) to speed up the
convergence of~\eqref{eq:approxexp} and achieve higher accuracy with
fewer function evaluations. Some ways variance reductions can be
accomplished are choosing a suitable \(\rho\) (importance sampling),
by transforming the integral onto another variable, a combination of
both approaches or by subdividing integration volume into several
sub-volumes of different size while keeping the sample size constant
in all sub-volumes (stratified sampling, a special case of importance
sampling). Combining ideas from importance sampling and stratified
sampling leads to the \vegas\ algorithm~\cite{Lepage:19781an}.

The convergence of~\eqref{eq:approxexp} is not dependent on the
dimensionality of the integration volume as opposed to many other
numerical integration algorithms (trapezoid rule, Simpsons rule) that
usually converge like \(N^{-\frac{k}{n}}\) with \(k\in\mathbb{N}\) as
opposed to \(N^{-\frac{k}{n}}\) with monte-carlo. Because integrals in
particle physics usually have a high dimensionality, monte-carlo
integration is suitable there. When implementing monte-carlo methods,
the random samples can be obtained through hardware or software random
number generators (RNGs). Most implementations utilize software RNGs
because supply pseudo-random numbers in a reproducible way, which
facilitates deniability and comparability.

\subsection{Naive Monte-Carlo Integration and Change of Variables}
\label{sec:naivechange}

The simplest choice for \(\rho\) is given
by~\eqref{eq:simplep}, the uniform distribution.

\begin{equation}
  \label{eq:simplep}
  \rho(\vb{x}) = \frac{1}{\int_{\Omega}1\dd{\vb{x'}}} =
  \frac{1}{\abs{\Omega}}
\end{equation}

With this distribution~\eqref{eq:approxexp}
becomes~\eqref{eq:approxexp-uniform}. In other words, \(I\) is just
the mean of \(f\) in \(\Omega\), henceforth
called \(\bar{f}\), multiplied with the volume.

\begin{equation}
  \label{eq:approxexp-uniform}
  \EX{\frac{F}{\Rho}} \approx
  \frac{\abs{\Omega}}{N}\sum_{i=1}^N f(\vb{x_i}) = \abs{\Omega}\cdot\bar{f}
\end{equation}

The variance \(\VAR{I}=\VAR{F/\Rho}\) is now given
by~\ref{eq:approxvar-I}. Note that the factor \(\abs{\Omega}\) gets
squared when approximating the integral by the sum.

\begin{equation}
  \label{eq:approxvar-I}
  \VAR{I} = \abs{\Omega}\int_\Omega f(\vb{x})^2 -
  I^2 \dd{\vb{x}} \equiv \abs{\Omega}\cdot\sigma_f^2 \approx
  \frac{\abs{\Omega}^2}{N-1}\sum_{i}\qty[f(\vb{x}_i) - \bar{f}]^2
\end{equation}

Applying this method to integrate
\(2\pi\sin(\theta)\cdot\dv{\sigma}{\Omega}\) (see~\eqref{eq:crossec})
over a \(\theta\) interval, equivalent to \(\eta\in [-2.5, 2.5]\) with
a target accuracy of \(\varepsilon=10^{-3}\) results in
\result{xs/python/xs_mc} with a sample size of
\result{xs/python/xs_mc_N}.

Changing variables and integrating~\eqref{eq:xs-eta} over \(\eta\)
with the same target accuracy yields~\result{xs/python/xs_mc_eta} with
a sample size of just~\result{xs/python/xs_mc_eta_N}. The dramatic
reduction in variance and sample size can be understood qualitatively
by studying~\ref{fig:xs-int-comp}. The differential cross section in
terms of \(\eta\)~(\ref{fig:xs-int-eta}) is less steep than the
differential cross section in terms of
\(\theta\)~(\ref{fig:xs-int-theta}) and takes on significant values
over most of the integration interval.

\begin{figure}[ht]
  \centering
  \begin{subfigure}[c]{.49\textwidth}
    \plot{xs/xs_integrand}
    \caption[\(2\pi\dv{\sigma}{\theta}\) with integration
    boundaries]{\label{fig:xs-int-theta} The integrand arising from
      differential cross section \(\dv{\sigma}{\theta}\) with the
      integration borders visualized as gray lines.}
  \end{subfigure}
  \begin{subfigure}[c]{.49\textwidth}
    \plot{xs/xs_integrand_eta}
    \caption[Differential cross section for \(\qqgg\) with integration
    boundaries]{\label{fig:xs-int-eta} The differential cross section
      \(\dv{\sigma}{\eta}\) (see~\ref{eq:xs-eta}) scaled by \(2\pi\)
      with the integration borders visualized as gray lines.}
  \end{subfigure}
  \caption{\label{fig:xs-int-comp} Comparison of two parametrisations
    of the differential cross section.}
\end{figure}

\subsection{Integration with \vegas}
\label{sec:mcintvegas}

Stratified sampling gives optimal results, when the variance in every
sub-volume is the same\cite{Lepage:19781an}. In importance sampling
the optimal probability distribution is given
by~\eqref{eq:optimalrho}, where \(f(\Omega) \geq 0\) is presumed
without loss of generality. When applying \vegas\ to multi dimensional
integrals, \ref{eq:optimalrho} is usually modified to factorize into
distributions for each variable.

\begin{equation}
  \label{eq:optimalrho}
  \rho(\vb{x}) = \frac{f(\vb{x})}{\int_\Omega f(\vb{y})\dd{y}}
\end{equation}

The idea behind \vegas\ is to subdivide \(\Omega\) into hypercubes,
define \(\rho\) as step-function on those hypercubes and iteratively
approximating~\eqref{eq:optimalrho}, instead of trying to minimize the
variance directly~\cite{Lepage:19781an}. In the end, the samples are
concentrated where \(f\) takes on the highest values and changes most
rapidly. This is done by subdividing the hypercubes into smaller
chunks, based on their contribution to the integral and then varying
the hypercube borders until all hypercubes contain the same number of
chunks. Note that no knowledge about the integrand is required. The
probability density used by \vegas\ is given in~\eqref{eq:vegasrho}
with \(K\) being the number of hypercubes and \(\Omega_i\) being the
hypercubes themselves.

\begin{equation}
  \label{eq:vegasrho}
  \rho(\vb{x}) = K\cdot
  \begin{cases}
    \frac{1}{\abs{\Omega_i}} & \vb{x}\in\Omega_{i\in\overline{1,K}} \\
    0 & \text{otherwise}
  \end{cases}
\end{equation}

\begin{figure}[ht]
  \centering \plot{xs/xs_integrand_vegas}
  \caption[\(2\pi\dv{\sigma}{\theta}\) scaled to increments found by
  \vegas\ ]{\label{fig:xs-int-vegas} The same integrand as
    in~\ref{fig:xs-int-theta} with \vegas-generated increments and
    weighting applied (\(f/\rho\)).}
\end{figure}

In one dimension the hypercubes become simple interval
increments. Applying \vegas\ to~\eqref{eq:crossec} with
\result{xs/python/xs_mc_θ_vegas_K} increments yields
\result{xs/python/xs_mc_θ_vegas} with
\result{xs/python/xs_mc_θ_vegas_N} samples. This result is comparable
with tho one obtained by parameter transformation
in~\ref{sec:naivechange}.  The sample count \(N\) is the number of
samples of the final integration and not the total number of
evaluations of \(f\) and is mentioned here, to illustrate the effect
of using stratified sampling. The resulting increments and the
weighted integrand \(f/\rho\) are depicted in~\ref{fig:xs-int-vegas},
along with the original integrand and it is intuitively clear, that
the variance is being reduced. Smaller increments correspond to higher
sample density and lower weights, flattening out the integrand.

Generally the result gets better with more increments, but at the cost
of more \vegas\ iterations. The intermediate values of those
iterations can be accumulated to improve the accuracy of the end
result.~\cite[197]{Lepage:19781an}
