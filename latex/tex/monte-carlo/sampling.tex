\section{Monte Carlo Sampling}%
\label{sec:mcsamp}

Drawing representative samples from a probability distribution (for
example a differential cross section) results in a set of
\emph{events}. This procedure is called sampling and produces the same
kind of data, that is gathered in experiments and from which one can
then calculate samples from the distribution of other observables
event-by-event without explicit transformation of the
distribution. Furthermore, these samples can be used as the basis for
more involved simulations (see \cref{chap:pheno}). Sampling shares
many characteristics with integration and thus the methods discussed
here use similar terminology and often inherit their fundamental ideas
from the integration methods.

Here the one-dimensional case is discussed.  Sampling a
multi-dimensional distribution is equivalent to sampling one
dimensional distributions by reducing the distribution itself to one
variable through integration over the remaining variables and then,
keeping the first variable fixed, sampling the other variables in a
likewise manner.

Consider a function \(f\colon x\in\Omega\mapsto\mathbb{R}_{\geq 0}\)
where \(\Omega = [0, 1]\) without loss of generality. Such a function
is proportional to a probability density \(\tilde{f}\). When \(X\) is
a uniformly distributed random variable on~\([0, 1]\) (which can be
easily generated), then a sample \({x_i}\) of this variable can be
transformed into a sample of \(Y\sim\tilde{f}\). Let \(x\) be a single
sample of \(X\). A sample \(y\) of \(Y\) can be obtained by
solving \cref{eq:takesample} for \(y\).
%
\begin{equation}
  \label{eq:takesample}
  \int_{0}^{y}f(x')\dd{x'} = x\cdot\int_0^1f(x')\dd{x'} = x\cdot A
\end{equation}
%
This can be shown by observing that, according to
\cref{eq:takesample}, the probability that \(y\in[y', y'+\dd{y}']\) is
the same as the probability that
\[x\in A^{-1}\qty[\int_{0}^{y'}f(x')\dd{x'},
\int_{0}^{y'+\dd{y}'}f(x')\dd{x'}]\] which is
\[A^{-1}\qty(\int_{0}^{y'+\dd{y}'}f(x')\dd{x'} -
  \int_{0}^{y'}f(x')\dd{x'}) = A^{-1} f(y')\dd{y}'\] and therefore
\(y\) is really distributed according to \(f/A=\tilde{f}\). If the
antiderivative \(F\) (and its inverse) of \(f\) is known, then the
solution of \cref{eq:takesample} is given by \cref{eq:solutionsamp}.
%
\begin{equation}
  \label{eq:solutionsamp}
  y = F^{-1}(x\cdot A + F(0))
\end{equation}
%
Note that \(F\) is always invertible because \(F\) is
increasing monotonically. Of course \(F\) and its inverse can be
obtained numerically or one can change variables to simplify.

\subsection{Importance Sampling and Hit or Miss}%
\label{sec:hitmiss}
If integrating \(f\) or inverting \(F\) is too expensive or a fully
\(f\)-agnostic method is desired, the problem can be reformulated by
introducing a positive function
\(g\colon x\in\Omega\mapsto\mathbb{R}_{\geq 0}\) with
\(\forall x\in\Omega\colon g(x)\geq f(x)\).

Observing \cref{eq:takesample2d} suggests, that one generates samples
which are distributed according to \(g/B\), where
\(B=\int_0^1g(x)\dd{x}\) and then accepts them with the
probability~\(f/g\), so that \(g\) cancels out. This is known as the
\emph{hit or miss} implementation of importance sampling.
%
\begin{equation}
  \label{eq:takesample2d}
  \int_{0}^{y}f(x')\dd{x'} =
  \int_{0}^{y}g(x')\cdot\frac{f(x')}{g(x')}\dd{x'}
  = \int_{0}^{y}g(x')\int_{0}^{\frac{f(x')}{g(x')}}\dd{z}\dd{x'}
\end{equation}
%
The thus obtained samples are then distributed according to \(f/B\)
and the total probability of accepting a sample, also called the
efficiency \(\mathfrak{e}\), is given by hat \cref{eq:impsampeff}
holds.
%
\begin{equation}
  \label{eq:impsampeff}
  \int_0^1\frac{g(x)}{B}\cdot\frac{f(x)}{g(x)}\dd{x} = \int_0^1\frac{f(x)}{B}\dd{x} = \frac{A}{B} = \mathfrak{e}\leq 1
\end{equation}
%
The closer the volumes enclosed by \(g\) and \(f\) are to each other,
higher is \(\mathfrak{e}\).

Choosing \(g\) like \cref{eq:primitiveg} and looking back at
\cref{eq:solutionsamp} yields \(y = x\cdot A\), so that the sampling
procedure simplifies to choosing random numbers \(x\in [0,1]\) and
accepting them with the probability \(f(x)/g(x)\). The efficiency of
this approach is related to how much \(f\) differs from
\(f_{\text{max}}\) which in turn related to the variance of
\(f\). Minimizing variance will therefore improve sampling
performance. The method can also be used in higher dimensions without
modification and has again been implemented and evaluated.
%
\begin{equation}
  \label{eq:primitiveg}
  g=\max_{x\in\Omega}f(x)=f_{\text{max}}
\end{equation}
%
\begin{figure}[ht]
  \centering
  \plot{xs_sampling/upper_bound}
  \caption{\label{fig:distcos} The distribution \cref{eq:distcos} and an upper bound of
    the form \(a + b\cdot x^2\).}
\end{figure}
%
Using the upper bound defined in \cref{eq:primitiveg} with the
distribution for \(\cos\theta\) derived from the differential cross
section \cref{eq:crossec} given in \cref{eq:distcos}
(\(\mathfrak{C}\) being a constant) results in a sampling efficiency
of~\result{xs/python/naive_th_samp}.
%
\begin{equation}
  \label{eq:distcos}
  f_{\cos\theta}(x=\cos\theta) = \mathfrak{C}\cdot\frac{1+x^2}{1-x^2}
\end{equation}
%
This very low efficiency stems from the fact, that \(f_{\cos\theta}\)
is a lot smaller than its maximum in most of the sampling interval.

Utilizing an upper bound of the form \(a + b\cdot x^2\) with \(a, b\)
constant improves the efficiency
to~\result{xs/python/tuned_th_samp}. The distribution, as well as the
upper bound are depicted in \cref{fig:distcos}.

\subsection{Importance Sampling through Change of Variables}%
\label{sec:importsamp}

When transforming \(f\) to a new variable \(y=y(x)\) one arrives at
\cref{eq:transff} and may reduce variance, analogous to
\cref{sec:naivechange}. Transforming the distribution in a beneficial
way in is an alternative method of performing \emph{importance
  sampling}.
%
\begin{equation}
  \label{eq:transff}
  \tilde{f} = f(x(y)) \cdot \dv{x(y)}{y}
\end{equation}
%
The optimal transformation would be the solution of \(y = F(x)\)
(\(F\) being the antiderivative), so that
\(f(x(y)) \cdot \dv{x(y)}{y} = 1\). But transforming \(f\) in this way
is the same as solving \cref{eq:takesample} which is a problem that
has been addressed in \cref{sec:hitmiss}. The difference here is, that
we restate the sampling problem in \(y\) space, which separates the
step of converting our \(y\) samples to \(x\) samples away from the
sampling process (see \cref{sec:obs}). Solving \cref{eq:takesample}
may now be easier, or applying the hit or miss method may be more
efficient.

When transforming the differential cross-section to the pseudo
rapidity \(\eta\) the efficiency of the hit or miss method rises
to~\result{xs/python/eta_eff} so applying this method in conjunction
with others is worthwhile (see \cref{fig:xs-int-comp}).

\subsection{Hit or Miss and \vegas}%
\label{sec:stratsamp}

Finding a suitable upper bound or variable transformation requires
effort and detail knowledge about the distribution and is hard to
automate. Revisiting the idea behind \cref{eq:takesample2d} but
looking at a probability density \(\rho > 0\) on \(\Omega\) leads to a
slight reformulation of the method discussed in
\cref{sec:hitmiss}. Note that without loss of generality one can again
choose \(\Omega = [0, 1]\).

Define \(h=\max_{x\in\Omega}f(x)/\rho(x)\), take a sample
\(\{\tilde{x}_i\}\sim\rho\) distributed according to \(\rho\) and
accept each sample point with the probability
\(f(x_i)/(\rho(x_i)\cdot h)\).  This is very similar to the procedure
described in \cref{sec:hitmiss} with \(g=\rho\cdot h\), but here the
step of generating samples distributed according to \(\rho\) is left
out as these samples are assumed to be available or can be obtained
with little effort (see below). The efficiency of this method is given
by \cref{eq:strateff}.
%
\begin{equation}
  \label{eq:strateff}
  \mathfrak{e} = \int_0^1\rho(x)\frac{f(x)}{\rho(x)\cdot h}\dd{x} = \frac{A}{h}
\end{equation}
%
It may seem startling that \(h\) determines the efficiency, because
\(h\) is a (global) maximum and \(A\) is an integral
but \cref{eq:hlessa} states that \(\mathfrak{e}\) is well-formed
(\(\mathfrak{e}\leq 1\)). Albeit \(h\) is determined through a single
point, being the maximum is a global property and there is also the
constraint \(\int_0^1\rho(x)\dd{x}=1\) to be considered.
%
\begin{equation}
  \label{eq:hlessa}
  A = \int_0^1\rho(x)\frac{f(x)}{\rho(x)}\dd{x} \leq
  \int_0^1\rho(x)\cdot h\dd{x} = h
\end{equation}
%
The closer \(h\) approaches \(A\) the better the efficiency gets. In
the optimal case \(\rho=f/A\) and thus \(h=A\) or
\(\mathfrak{e} = 1\). This distribution can be approximated in the way
discussed in \cref{sec:mcintvegas} by using the hypercubes found
by~\vegas\ and simply generating the same number of uniformly
distributed samples in each hypercube. The distribution \(\rho\) takes
on the form \cref{eq:vegasrho}. The effect of this approach is
visualized in \cref{fig:vegasdist} and the resulting sampling
efficiency \result{xs/python/strat_th_samp} (using
\result{xs/python/vegas_samp_num_increments} increments) is a great
improvement over the hit or miss method in \cref{sec:hitmiss} and even
surpasses the variable transformation to \(\eta\). By using more
increments better efficiencies can be achieved, although the run-time
of \vegas\ increases. The advantage of \vegas\ in this situation is,
that the computation of the increments has to be done only once and
can be reused. Furthermore, no special knowledge about the
distribution \(f\) is required.
%
\begin{figure}[ht]
  \centering
  \begin{subfigure}{.49\textwidth}
    \plot{xs_sampling/vegas_strat_dist}
    \caption[The distribution for \(\cos\theta\), derived from the
    differential cross-section and the \vegas-weighted
    distribution]{\label{fig:vegasdist} The distribution for
      \(\cos\theta\) (see \cref{eq:distcos}) and the \vegas-weighted
      distribution.}
  \end{subfigure}
  \begin{subfigure}{.49\textwidth}
    \plot{xs_sampling/vegas_rho}
    \caption[The weighting distribution generated by
    \vegas.]{\label{fig:vegasrho} The weighting distribution generated
      by \vegas. It is clear, that it closely follows the original
      distribution \cref{eq:distcos}.}
  \end{subfigure}
  \caption{\label{fig:vegas-weighting} \vegas-weighted distribution
    and weighting distribution.}
\end{figure}
%


\subsection{Stratified Sampling}
\label{sec:stratsamp-real}

Yet another approach is to subdivide the sampling volume \(\Omega\)
into \(K\) sub-volumes \(\Omega_i\subset\Omega\) and then take a
number of samples from each volume proportional to the integral of the
function \(f\) in that volume. In the context of sampling of
distributions, this is known as \emph{stratified
  sampling}. The advantage of this method is, that it is now possible
to optimize the sampling in each sub-volume independently.

Let \(N\) be the total sample count (\(N\gg 1\)),
\(A_i = \int_{\Omega_i}f(x)\dd{x}\) and \(A=\sum_iA_i\).
Then we can calculate the total efficiency of taking \(N_i=A_i/A \cdot N\)
samples in each is then given by \cref{eq:rstrateff}.
%
\begin{equation}
  \label{eq:rstrateff}
  \mathfrak{e} = \frac{\sum_i N_i}{\sum_i N_i/\mathfrak{e}_i} =
  \frac{\sum_i A_i/A\cdot N}{\sum_i A_i/A\cdot N/ \mathfrak{e}_i} = \frac{\sum_i A_i}{\sum_i A_i/\mathfrak{e}_i}
\end{equation}
%
In the case when all \(\mathfrak{e}_i\) are the same, the total
efficiency is the same as the individual efficiencies. In the case of
one efficiency being much smaller then the others, this efficiency
dominates the overall efficiency (assuming somewhat similar
\(A_i\)). So in general one should optimize so that the individual
efficiencies are roughly the same. Using the \(\Omega_i\) generated by
\vegas\ has the advantage, that this requirement can be approximated
and the \(A_i\) have already been obtained by \vegas. The technical
difficulty in the implementation is the way in which sample points get
distributed among the sub-volumes. The pitfall here is that the
\(A_i\) (and the upper bounds for the hit-or-miss method) have to be
determined increasingly accurate with growing sample size.

This method will be applied to multi-dimensional sampling in
\cref{sec:pdf_results}.

\subsection{Observables}%
\label{sec:obs}

Having obtained a sample of a distribution, distributions of other
observables can be calculated from those samples without having to
transform the distribution into new variables. This is due to the
discrete nature of the samples. Suppose there is an observable
\(\gamma\colon\Omega\mapsto\mathbb{R}\). Now to take a sample
\(\{x_i\}\) of \(\gamma\) we sample \(f\) and convert the sample
values by simply applying \(\gamma\), where \(f\) is as defined in
\cref{sec:mcsamp}. This is equivalent to substituting
\(y=\gamma^{-1}(z)\) in \cref{eq:takesample} and solving for \(z\).

The probability that \(z\in[z', z'+\dd{z'}]\) now is the same as the
probability that
\[\displaystyle x\in
  A^{-1}\qty[\int_{0}^{\gamma^{-1}(z')}f(x')\dd{x'},
  \int_{0}^{\gamma^{-1}(z')+\partial_z(\gamma^{-1})(z')\dd{z'}}f(x')\dd{x'}]\]
which is
\(A^{-1}\cdot f(\gamma^{-1}(z'))\cdot
(\partial_z\gamma^{-1})(z')\dd{z'}\). That is the same result, as if
the distribution had been transformed by multiplying the appropriate
Jacobian.

Using the distribution \cref{eq:distcos} for the variable
\(\cos\theta\) and choosing the polar angle \(\varphi\) uniformly
random, a sample of 4-momenta can be generated and histograms
observables can be drawn.

The observables considered here are the transverse momentum \(\pt\)
and the pseudo rapidity \(\eta\) which can be computed from 4-momentum
as described in \cref{eq:observables}.
%
\begin{align}
  \label{eq:observables}
  \pt &= \sqrt{(p_x)^2+(p_y)^2} & \eta &=
                                         \arcosh\qty(\frac{\abs{\vb{p}}}{\pt})\cdot\sign(p_z)
\end{align}
%
The histograms in \cref{fig:histos} have been created by generating
\result{xs/python/4imp-sample-size} samples. \Cref{fig:histos} also
contains reference histograms created by generating events with
\sherpa\ and analyzing them with the \rivet\
toolkit~\cite{Bierlich:2019rhm}. The utilized analysis can be found
in \cref{sec:simpdiphotriv}.
%
\begin{figure}[ht]
  \centering \plot{xs_sampling/diff_xs_p_t}
  \caption{\label{fig:diff-xs-pt} The differential cross section
    transformed to \(\pt\).}
\end{figure}
%
\begin{figure}[p]
  \centering

  \begin{subfigure}[b]{\textwidth}
    \centering \plot{xs_sampling/histo_sherpa_eta}
    \caption{\label{fig:histeta} Histogram of the pseudo-rapidity
      (\(\eta\)) distribution.}
  \end{subfigure}
  \begin{subfigure}[b]{\textwidth}
    \centering \plot{xs_sampling/histo_sherpa_pt}
    \caption{\label{fig:histpt} Histogram of the transverse momentum
      (\(\pt\)) distribution.}
  \end{subfigure}
  \caption{\label{fig:histos} Histograms of observables, generated
    from a sample of 4-momenta and normalized to unity. The plots
    include histograms generated by \sherpa\ and \rivet.}
\end{figure}
%
Where \cref{fig:histeta} shows clear resemblance of
\cref{fig:xs-int-eta}, the peak and the rise before this peak in
\cref{fig:histpt} around \(\pt=\SI{100}{\giga\electronvolt}\) seems
surprising and opens the possibility for the production of hard
transverse photons. When transforming the differential cross section
to \(\pt\) it can be seen in \cref{fig:diff-xs-pt} that there really
is a singularity at \(\pt = \ecm\), due to a term
\(1/\sqrt{1-(2\cdot \pt/\ecm)^2}\) stemming from the Jacobian
determinant.  This singularity will vanish once considering a more
realistic process (see \cref{chap:pdf}). Furthermore the histograms
\cref{fig:histeta,fig:histpt} are consistent with their
\rivet-generated counterparts and are therefore considered valid.

%%% Local Variables:
%%% mode: latex
%%% TeX-master: "../../document"
%%% End:
