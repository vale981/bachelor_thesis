%%% Local Variables: ***
%%% mode: latex ***
%%% TeX-master: "../../document.tex"  ***
%%% End: ***

\section{Monte-Carlo Sampling}%
\label{sec:mcsamp}

Drawing representative samples from a probability distribution (for
example a differential cross section) results in a set of
\emph{events}, the same kind of data, that is gathered in experiments
and from which one can the calculate samples from the distribution of
other observables without explicit transformation of the
distribution. Here the one-dimensional case is discussed. The general
case follows by sampling the dimensions sequentially.

Consider a function \(f\colon x\in\Omega\mapsto\mathbb{R}_{\geq 0}\)
where \(\Omega = [0, 1]\) without loss of generality. Such a function
is proportional to a probability density \(\tilde{f}\). When \(X\) is
a uniformly distributed random variable on~\([0, 1]\) (which can be
easily generated) then a sample \({x_i}\) of this variable can be
transformed into a sample of \(Y\sim\tilde{f}\). Let \(x\) be a single
sample of \(X\), then a sample \(y\) of \(Y\) can be obtained by
solving~\eqref{eq:takesample} for \(y\).

\begin{equation}
  \label{eq:takesample}
  \int_{0}^{y}f(x')\dd{x'} = x\cdot\int_0^1f(x')\dd{x'} = x\cdot A
\end{equation}

This can be shown by observing that, according
to~\eqref{eq:takesample}, the probability that
\(y\in[y', y'+\dd{y}']\) is the same as the probability that
\(x\in A^{-1}\qty[\int_{0}^{y'}f(x')\dd{x'},
\int_{0}^{y'+\dd{y}'}f(x')\dd{x'}]\) which is
\(A^{-1}\qty(\int_{0}^{y'+\dd{y}'}f(x')\dd{x'} -
\int_{0}^{y'}f(x')\dd{x'}) = A^{-1} f(y')\dd{y}'\). So \(y\) is really
distributed according to \(f/A\).

If the antiderivative \(F\) of is known, then the solution
of~\eqref{eq:takesample} is given by~\eqref{eq:solutionsamp}.

\begin{equation}
  \label{eq:solutionsamp}
  y = F^{-1}(x\cdot A + F(0))
\end{equation}

Note that \(F\) is always invertible because \(F\) is
increasing monotonically. Of course \(F\) and its inverse can be
obtained numerically or one can change variables to simplify.

\subsection{Hit or Miss}%
\label{sec:hitmiss}
If integrating \(f\) and/or inverting \(F\) is too expensive or a
fully \(f\)-agnostic method is desired, the problem can be
reformulated by introducing a positive function
\(g\colon x\in\Omega\mapsto\mathbb{R}_{\geq 0}\) with
\(\forall x\in\Omega\colon g(x)\geq f(x)\).

Observing~\eqref{eq:takesample2d} suggests, that one generates samples
which are distributed according to \(g/B\), where
\(B=\int_0^1g(x)\dd{x}\) and then accepts them with the
probability~\(f/g\), so that \(g\) cancels out. This method is called
``hit or miss''.

\begin{equation}
  \label{eq:takesample2d}
  \int_{0}^{y}f(x')\dd{x'} =
  \int_{0}^{y}g(x')\cdot\frac{f(x')}{g(x')}\dd{x'}
  = \int_{0}^{y}g(x')\int_{0}^{\frac{f(x')}{g(x')}}\dd{z}\dd{x'}
\end{equation}

The thus obtained samples are then distributed according to \(f/B\)
and the total probability of accepting a sample (efficiency
\(\mathfrak{e}\)) is given by hat~\eqref{eq:impsampeff} holds.

\begin{equation}
  \label{eq:impsampeff}
  \int_0^1\frac{g(x)}{B}\cdot\frac{f(x)}{g(x)}\dd{x} = \int_0^1\frac{f(x)}{B}\dd{x} = \frac{A}{B} = \mathfrak{e}\leq 1
\end{equation}

The closer the volumes enclosed by \(g\) and \(f\) are to each other,
higher is \(\mathfrak{e}\).

Choosing \(g\) like~\eqref{eq:primitiveg} and looking back
at~\eqref{eq:solutionsamp} yields \(y = x\cdot A\), so that the
sampling procedure simplifies to choosing random numbers
\(x\in [0,1]\) and accepting them with the probability
\(f(x)/g(x)\). The efficiency of this approach is related to how much
\(f\) differs from \(f_{\text{max}}\) which in turn related to the
variance of \(f\). Minimizing variance will therefore improve sampling
performance.

\begin{equation}
  \label{eq:primitiveg}
  g=\max_{x\in\Omega}f(x)=f_{\text{max}}
\end{equation}

Using the upper bound defined in~\eqref{eq:primitiveg} with the
distribution for \(\cos\theta\) derived from the differential cross
section~\eqref{eq:crossec} given in~\eqref{eq:distcos}
(\(\mathfrak{C}\) being a constant) results in a sampling efficiency
of~\result{xs/python/naive_th_samp}.

\begin{equation}
  \label{eq:distcos}
  f_{\cos\theta}(x=\cos\theta) = \mathfrak{C}\cdot\frac{1+x^2}{1-x^2}
\end{equation}

This very low efficiency stems from the fact, that \(f_{\cos\theta}\)
is a lot smaller than its upper bound for most of the sampling
interval.

\begin{wrapfigure}[15]{l}{.5\textwidth}
  \plot{xs_sampling/upper_bound}
  \caption{\label{fig:distcos} The distribution~\eqref{eq:distcos} and an upper bound of
    the form \(a + b\cdot x^2\).}
\end{wrapfigure}

Utilizing an upper bound of the form \(a + b\cdot x^2\) with \(a, b\)
constant improves the efficiency
to~\result{xs/python/tuned_th_samp}. The distribution, as well as the
upper bound are depicted in~\ref{fig:distcos}

When transforming the differential cross-section to the pseudo
rapidity \(\eta\) the efficiency of the hit or miss method rises
to~\result{xs/python/eta_eff}, again due to the decrease in variance.

\subsection{Stratified Sampling}%
\label{sec:stratsamp}
Finding a suitable upper bound or variable transform requires effort
and detail knowledge about the distribution and is hard to
automate\footnote{Sherpa does in fact do this by looking at the
  propagators in the matrix elements.}.  Revisiting the idea
behind~\eqref{eq:takesample2d} but looking at probability density
\(\rho\) on \(\Omega\) leads to a slight reformulation of the method
discussed in~\ref{sec:hitmiss}. Note that without loss of generality
one can again choose \(\Omega = [0, 1]\).

Define \(h=\max_{x\in\Omega}f(x)/\rho(x)\), take a sample
\(\{\tilde{x}_i\}\sim\rho\) distributed according to \(\rho\) and
accept each sample point with the probability
\(f(x_i)/(\rho(x_i)\cdot h)\).  This is very similar to the procedure
described in~\ref{sec:hitmiss} with \(g=\rho\cdot h\), but here the
step of generating samples distributed according to \(\rho\) is left
out.

The important benefit of this method is, that step of generating
samples according to some other function \(g\) is no longer
necessary. This is useful when samples of \(\rho\) can be obtained
with little effort (see below). The efficiency of this method is given
by~\eqref{eq:strateff}.

\begin{equation}
  \label{eq:strateff}
  \mathfrak{e} = \int_0^1\rho(x)\frac{f(x)}{\rho(x)\cdot h}\dd{x} = \frac{A}{h}
\end{equation}

It may seem startling that \(h\) determines the efficiency, because
\(h\) is a (global) maximum and \(A\) is an integral
but~\eqref{eq:hlessa} states that \(\mathfrak{e}\) is well-formed
(\(\mathfrak{e}\leq 1\)). Albeit \(h\) is determined through a single
point, being the maximum is a global property and there is also the
constraint \(\int_0^1\rho(x)\dd{x}=1\) to be considered.

\begin{equation}
  \label{eq:hlessa}
  A = \int_0^1\rho(x)\frac{f(x)}{\rho(x)}\dd{x} \leq
  \int_0^1\rho(x)\cdot h\dd{x} = h
\end{equation}

The closer \(h\) approaches \(A\) the better the efficiency gets. In
the optimal case \(\rho=f/A\) and thus \(h=A\) or
\(\mathfrak{e} = 1\). Now this distribution can be approximated in the
way discussed in~\ref{sec:mcintvegas} by using the hypercubes found
by~\vegas and simply generating the same number of uniformly
distributed samples in each hypercube (stratified sampling). The
distribution \(\rho\) takes on the form~\eqref{eq:vegasrho}. The
effect of this approach is visualized in~\ref{fig:vegasdist} and the
resulting sampling efficiency \result{xs/python/strat_th_samp} (using
\result{xs/python/vegas_samp_num_increments} increments) is a great
improvement over the hit or miss method in~\ref{sec:hitmiss}. By using
more increments better efficiencies can be achieved, although the
run-time of \vegas\ increases. The advantage of \vegas\ in this
situation is, that the computation of the increments has to be done
only once and can be reused. Furthermore, no special knowledge about
the distribution \(f\) is required.

\begin{figure}[ht]
  \centering
  \begin{subfigure}{.49\textwidth}
    \plot{xs_sampling/vegas_strat_dist}
    \caption[The distribution for \(\cos\theta\), derived from the
    differential cross-section and the \vegas-weighted
    distribution]{\label{fig:vegasdist} The distribution for
      \(\cos\theta\) (see~\eqref{eq:distcos}) and the \vegas-weighted
      distribution. The inc It is intuitively clear, how variance is
      being reduced.}
  \end{subfigure}
  \begin{subfigure}{.49\textwidth}
    \plot{xs_sampling/vegas_rho}
    \caption[The weighting distribution generated by
    \vegas.]{\label{fig:vegasrho} The weighting distribution generated
      by \vegas. It is clear, that it closely follows the original
      distribution~\eqref{eq:distcos}.}
  \end{subfigure}
  \caption{\label{fig:vegas-weighting} \vegas-weighted distribution
    and weighting distribution.}
\end{figure}

\subsection{Observables}
\label{sec:obs}

Having obtained a sample of a distribution, distributions of other
observables can be calculated from those samples without having to
transform the distribution into new variables. This is due to the
discrete nature of the samples. Suppose there is an observable
\(\gamma\colon\Omega\mapsto\mathbb{R}\). Now to take a sample
\(\{x_i\}\) of \(\gamma\) we sample \(f\)\footnote{As defined
  in~\ref{sec:mcsamp}.} and convert the sample values by simply
applying \(\gamma\). This is equivalent to
substituting \(y=\gamma^{-1}(z)\) in~\eqref{eq:takesample} and solving
for \(z\).

The probability that \(z\in[z', z'+\dd{z'}]\) now is the same as the
probability that
\[\displaystyle x\in
  A^{-1}\qty[\int_{0}^{\gamma^{-1}(z')}f(x')\dd{x'},
  \int_{0}^{\gamma^{-1}(z')+\partial_z(\gamma^{-1})(z')\dd{z'}}f(x')\dd{x'}]\]
which is
\(A^{-1}\cdot f(\gamma^{-1}(z'))\cdot (\partial_z\gamma^{-1})(z')\dd{z'}\). That
is the same result, as if the distribution had been transformed by
multiplying the appropriate Jacobian.

Using the distribution~\eqref{eq:distcos} for the variable
\(\cos\theta\) and choosing the polar angle \(\varphi\) uniformly
random, a sample of 4-momenta can be generated and histograms
observables can be drawn.

The observables considered here are the transverse momentum \(\pt\)
and the pseudo rapidity \(\eta\) which can be computed from 4-momentum
as described in~\eqref{eq:observables}.

\begin{align}
  \label{eq:observables}
  \pt &= \sqrt{(p_1)^2+(p_2)^2} & \eta &=
                                         \frac{\abs{\vb{p}}}{\pt}\cdot\sign(p^3)
\end{align}

The histograms in~\ref{fig:histos} have been created by generating
\result{xs/python/4imp-sample-size} samples. \Ref{fig:histos} also
contains reference histograms created by generating events with
\sherpa\ and analyzing them with the \rivet\
toolkit~\cite{Bierlich:2019rhm}. The utilized analysis can be found
in~\ref{sec:simpdiphotriv}.

\begin{figure}[hb]
  \centering \plot{xs_sampling/diff_xs_p_t}
  \caption{\label{fig:diff-xs-pt} The differential cross section
    transformed to \(\pt\).}
\end{figure}

\begin{figure}[p]
  \centering

  \begin{subfigure}[b]{\textwidth}
    \centering \plot{xs_sampling/histo_sherpa_eta}
    \caption{\label{fig:histeta} Histogram of the pseudo-rapidity
      (\(\eta\)) distribution.}
  \end{subfigure}
  \begin{subfigure}[b]{\textwidth}
    \centering \plot{xs_sampling/histo_sherpa_pt}
    \caption{\label{fig:histpt} Histogram of the transverse momentum
      (\(\pt\)) distribution.}
  \end{subfigure}
  \caption{\label{fig:histos} Histograms of observables, generated
    from a sample of 4-momenta and normalized to unity. The plots
    include histograms generated by \sherpa\ and \rivet.}
\end{figure}

Where~\ref{fig:histeta} shows clear resemblance
of~\ref{fig:xs-int-eta}, the sharp peak in~\ref{fig:histpt} around
\(\pt=\SI{100}{\giga\electronvolt}\) seems surprising. When
transforming the differential cross section to \(\pt\) it can be seen
in~\ref{fig:diff-xs-pt} that there really is a singularity at
\(\pt =\abs{\vb{p}}\). Furthermore the histograms~\ref{fig:histeta}
and~\ref{fig:histpt} are consistent with their \rivet-generated
counterparts and are therefore considered valid.
