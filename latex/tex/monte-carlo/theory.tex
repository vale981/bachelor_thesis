%%% Local Variables: ***
%%% mode: latex ***
%%% TeX-master: "../../document.tex"  ***
%%% End: ***

\section{Monte-Carlo Integration}
\label{sec:mctheory}

Consider a function
\(f: \vb{x}\in\Omega\subset\mathbb{R}^n\mapsto\mathbb{R}\) and a
probability density on \(\Omega\)
\(\rho: \vb{x}\in\mathbb{R}^n\mapsto\mathbb{R}_{\geq 0}\) with
\(\int_{\Omega}\rho(\vb{x})\dd{\vb{x}} = 1\).  By multiplying \(f\)
with a \(1\) in the fashion of~\eqref{eq:baseintegral}, the Integral
of \(f\) over \(\Omega\) can be interpreted as the expected value
\(\EX{F/\Rho}\) of the random variable \(F/\Rho\)
under the distribution \(\rho\).

\begin{equation}
  \label{eq:baseintegral}
  I = \int_\Omega f(\vb{x}) \dd{\vb{x}} = \int_\Omega
  \qty[\frac{f(\vb{x})}{\rho(\vb{x})}] \rho(\vb{x}) \dd{\vb{x}} = \EX{\frac{F}{\Rho}}
\end{equation}

The expected value \(\EX{F/\Rho}\) can be approximate by taking the
mean of \(F/\Rho\) with \(N\) finite samples
\(\{\vb{x}_i\}_{i\in\overline{1,N}}\sim\rho\) (distributed according to
\(\rho\)), where \(N\) is usually a very large integer.

\begin{equation}
  \label{eq:approxexp}
  \EX{\frac{F}{\Rho}} \approx
  \frac{1}{N}\sum_{i=1}^N\frac{f(\vb{x_i})}{\rho(\vb{x_i})}
  \xrightarrow{N\rightarrow\infty} I
\end{equation}

The convergence of~\eqref{eq:approxexp} is due to the nature of the
expected value~\eqref{eq:evalue-mean} and
variance~\eqref{eq:variance-mean} of the mean
\(\overline{X} = \frac{1}{N}\sum_i X_i\) of \(N\) uncorrelated random
variables \(\{X_i\}_{i\in\overline{1,N}}\) with the same distribution,
expected value \(\EX{X_i}=\mathbb{E}\) and variance
\(\sigma_i^2 = \sigma^2\).

\begin{align}
  \EX{\overline{X}} = \frac{1}{N}\sum_i\EX{X_i} = \mathbb{E} \label{eq:evalue-mean}\\
  \sigma^2_{\overline{X}} = \sum_i\frac{\sigma_i^2}{N^2} =
                            \frac{\sigma^2}{N}  \label{eq:variance-mean}
\end{align}

Evidently \(\frac{\sigma^2}{N}\xrightarrow{N\rightarrow\infty} 0\)
thus the~\eqref{eq:approxexp} really converges to \(I\). For finite
\(N\) value of~\eqref{eq:approxexp} varies around \(I\) with the
variance \(\VAR{F/\Rho}\cdot N^{-1}\) as in~\eqref{eq:varI}.

\begin{align}
  \VAR{\frac{F}{\Rho}} &= \int_\Omega \qty[I -
  \frac{f(\vb{x})}{\rho(\vb{x})}]^2 \rho({\vb{x}}) \dd{\vb{x}} =
  \int_\Omega \qty[\qty(\frac{f(\vb{x})}{\rho(\vb{x})})^2 -
  I^2]\rho({\vb{x}}) \dd{\vb{x}}   \label{eq:varI}
 \\
  &\approx \frac{1}{N - 1}\sum_i \qty[I -
  \frac{f(\vb{x_i})}{\rho(\vb{x_i})}]^2  \label{eq:varI-approx}
\end{align}

The name of the game is thus to reduce \(\VAR{F/\Rho}\) to speed up
the convergence of~\eqref{eq:approxexp} and achieve higher accuracy
with fewer function evaluations.

The simplest choice for \(\rho\) is clearly given
by~\eqref{eq:simplep}, the uniform distribution.

\begin{equation}
  \label{eq:simplep}
  \rho(\vb{x}) = \frac{1}{\int_{\Omega}1\dd{\vb{x'}}} =
  \frac{1}{\abs{\Omega}}
\end{equation}

With this distribution~\eqref{eq:approxexp}
becomes~\eqref{eq:approxexp-uniform}. In other words, \(I\) is just
the mean of \(f\) in \(\Omega\), henceforth
called \(\bar{f}\), multiplied with the volume.

\begin{equation}
  \label{eq:approxexp-uniform}
  \EX{\frac{F}{\Rho}} \approx
  \frac{\abs{\Omega}}{N}\sum_{i=1}^N f(\vb{x_i}) = \abs{\Omega}\cdot\bar{f}
\end{equation}

The variance \(\VAR{I}=\VAR{F/\Rho}\) is now given
by~\ref{eq:approxvar-I}. Note that the factor \(\abs{\omega}\) gets
squared when approximating the integral to the sum.

\begin{equation}
  \label{eq:approxvar-I}
  \VAR{I} = \abs{\Omega}\int_\Omega f(\vb{x})^2 -
  I^2 \dd{\vb{x}} \equiv \abs{\Omega}\cdot\sigma_f^2 \approx
  \frac{\abs{\Omega}^2}{N-1}\sum_{i}\qty[f(\vb{x}_i) - \bar{f}]^2
\end{equation}
