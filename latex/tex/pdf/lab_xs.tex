\section{Kinematics and Cross-Section of the Diphoton Process}%
\label{sec:lab_xs}

To utilize \cref{eq:pdf-xs} for modeling the~\(\ppgg\) process and to
generate event samples the results of \cref{chap:qqgg} have to be
transformed into the center of momentum frame of the colliding
protons. Quantities in the center of momentum frame of the partons
will be starred (like \(x^\ast\)).

Let \(E_p\) be the beam energy of the proton beams and \(x_1, x_2\)
the momentum fractions of the partons in their respective protons.
Choosing the frame where the proton momenta are parallel to the
\(z\)-Axis and neglecting the mass of the protons their momenta are
\(\bar{p}_{1,2} = E_P \mqty(1 & 0 & 0 & \pm1)\). The quark momenta are
simply \(p_{1,2}=x_{1,2}\cdot\bar{p}_{1,2}\) and their invariant mass
is thus given by \cref{eq:ecm_partons}.

\begin{equation}
  \label{eq:ecm_partons}
  \ecm = \sqrt{4x_1x_2}\cdot E_P
\end{equation}

The center of momentum frame moves with a velocity
\(\beta = (x_1-x_2)/(x_1+x_2)\). Because (pseudo-)rapidities are
additive, it is simplest to work with cross section in terms of the
pseudo-rapidity.  The pseudo-rapidity of a final state photon in the
c.m.-frame of the partons is therefore given
by \cref{eq:rap_phot_lf}.

\begin{equation}
  \label{eq:rap_phot_lf}
  \eta^\ast = \eta - w = \eta - \artanh(\beta)
\end{equation}

Because \(\dd{\eta^\ast}/{\dd{\eta}} = 1\) the cross section
from \cref{eq:xs-eta} becomes \cref{eq:xs-eta-lab}, where \(\eta\) is
the pseudo-rapidity one photon.

\begin{equation}
  \label{eq:xs-eta-lab}
  \dv{\sigma}{\eta} = 2\pi\cdot\frac{\alpha^2Z^4}{24 E_p^2
    x_1x_2}\cdot\qty(\tanh(\eta - w)^2 + 1)
\end{equation}

\begin{wrapfigure}{R}{0.4\textwidth}
\centering
\begin{tikzpicture}
  \coordinate (origin) at (0,0);

  \draw[Latex-] (origin) --  node [midway, above] {\(x_1\)} (-2,0) node[left] {\(p_1\)};
  \draw[Latex-] (origin) -- node [midway, below] {\(x_2\)} (2,0) coordinate (p4) node[right] {\(p_2\)};
  \draw[-Latex,rotate=40] (origin) -- (2,0) coordinate (p2) node[right] {\(p_3\)};
  \draw[-Latex,rotate=70] (origin) -- (-2,0) node[left] {\(p_4\)};
  \draw[fill=black] (origin) circle (.03);

  \draw pic["$\theta$", draw=black, <->, angle eccentricity=1.2, angle radius=1.5cm] {angle=p4--origin--p2};
\end{tikzpicture}
\end{wrapfigure}
The 4-momenta of the final state photons can be obtained by explicit
Lorentz transformation and are listed in \cref{eq:lab-momenta-fs}
where \(\theta\) is the azimuth angle of the ``first'' final state
photon.  These relation are easily transcribed to \(\eta\) dependence
by using the identity \(\cos(\theta) = \tanh(\eta)\).


\begin{align}
  \label{eq:lab-momenta-fs}
  p_3 &= \frac{2x_1x_2E_p}{x_1+x_2 - (x_1 - x_2)\cdot\cos(\theta)}\cdot
  \mqty(1 \\ 0 \\ \sin(\theta) \\ \cos(\theta)) \\
  p_4 &= \frac{E_p}{x_1+x_2 - (x_1 - x_2)\cdot\cos(\theta)}\cdot
  \mqty((x_1^2+x_2^2) - (x_1^2-x_2^2)\cos(\theta) \\ 0 \\ -2x_1x_2\sin(\theta) \\ (x_1^2-x_2^2) - (x_1^2+x_2^2)\cos(\theta))
\end{align}

%%% Local Variables:
%%% mode: latex
%%% TeX-master: "../../document"
%%% End:
