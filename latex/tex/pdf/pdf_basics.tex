\section{Parton Density Functions}%
\label{sec:pdf_basics}

Parton Density Functions encode, restricting considerations to leading
order, the probability to \emph{encounter} a constituent parton (quark
or gluon) of a hadron with a certain momentum fraction \(x\) at a
certain factorization scale \(Q^2\). PDFs are normalized according to
\cref{eq:pdf-norm}, where the sum runs over all partons.

\begin{equation}
  \label{eq:pdf-norm}
  \sum_i\int_0^1x\cdot f_i\qty(x;Q^2) \dd{x} = 1
\end{equation}

More precisely \({f_i}\) denotes a PDF set, which is referred to
simply as PDF in the following.  PDFs can not be derived from first
principles (at the moment) and have to be determined experimentally
for low \(Q^2\) and are evolved to higher \(Q^2\) through the
\emph{DGLAP} equations~\cite{altarelli:1977af} at different orders of
perturbation theory.  In deep inelastic scattering \(Q^2\) is just the
negative over the momentum transfer \(-q^2\). For more complicated
processes \(Q^2\) has to be chosen in a way that reflects the
\emph{momentum resolution} of the process. If the perturbation series
behind the process would be expanded to the exact solution, the
dependence on the factorization scale vanishes. In lower orders, one
has to choose the scale in a \emph{physically
  meaningful}\footnote{That means: not in an arbitrary way.} way,
which reflects characteristics of the process~\cite{altarelli:1977af}.

In the case of \(\qqgg\) the mean of the Mandelstam variables \(\hat{t}\)
and \(\hat{u}\), which is equal to \(\hat{s}/2\), can be used. This
choice is lorentz-invariant and reflects the s/u-channel nature of the
process.

The (differential) hadronic cross section for scattering of two
partons in equal hadrons is given in \cref{eq:pdf-xs}. Here \(i,j\)
are the partons participating in a scattering process with the cross
section \(\hat{\sigma}_{ij}\). Usually this cross section depends on
the kinematics and thus the momentum fractions and the factorization
scale\footnote{More appropriately: The factorization scale depends on
  the process. So \(\sigma\qty(Q^2)\) is just a symbol for that
  relation.}.

\begin{equation}
  \label{eq:pdf-xs}
  \sigma = \int f_i\qty(x;Q^2) f_j\qty(x;Q^2) \hat{\sigma}_{ij}\qty(x_1,
  x_2, Q^2)\dd{x_1}\dd{x_2}
\end{equation}

Summing \cref{eq:pdf-xs} over all partons in the hadron gives
the total scattering cross section for the hadron.

%%% Local Variables:
%%% mode: latex
%%% TeX-master: "../../document"
%%% End:
