\section{Implementation and Results}%
\label{sec:pdf_results}

The considerations of \cref{sec:pdf_basics,sec:lab_xs} can now be
applied to obtain a cross section and histograms of observables for
the scattering of two protons into two photons. Because the PDF is not
available in closed form, event generation is the only viable way to
obtain distributions of observables and verify theory against
experiment, even with this simple leading-order process.

The integrand in \cref{eq:pdf-xs} can be concertized into
\cref{eq:weighteddist}, where \(q\) runs over all quarks (except the
top quark). The sum has been symmetized, otherwise a double sum with
\(q\) and \(\bar{q}\) would have been necessary. The choice of \(Q^2\)
is justified in \cref{sec:pdf_basics} and formulated in
\cref{eq:q2-explicit}.

\begin{gather}
  \label{eq:weighteddist}
  \frac{\dd[3]{\sigma}}{\dd{\eta}\dd{x_1}\dd{x_2}} =
  \sum_q \qty[f_q\qty(x_1;Q^2) f_{\bar{q}}\qty(x_2;Q^2) + f_q\qty(x_2;Q^2) f_{\bar{q}}\qty(x_1;Q^2)] \dv{\sigma(x_1,
    x_2, Z_q)}{\eta} \\
  \label{eq:q2-explicit}
  Q^2 = 2x_1x_2E_p^2
\end{gather}

The PDF set being used in the following has been fitted (and
developed) at leading order and is the central member of the PDF set
\verb|NNPDF31_lo_as_0118| provided by \emph{NNPDF} collaboration and
accessed through the \lhapdf\
library~\cite{NNPDF:2017pd}\cite{Buckley:2015lh}.

\subsection{Cross Section}%
\label{sec:ppxs}

The distribution \cref{eq:weighteddist} can now be integrated to
obtain a total cross-section as described in \cref{sec:mcint}.  For
the numeric analysis a proton beam energy of
\result{xs/python/pdf/e_proton} has been chosen, in accordance to
\lhc{} beam energies. As for the cuts, \result{xs/python/pdf/eta} and
\result{xs/python/pdf/min_pT} have been set.  Integrating
\cref{eq:weighteddist} with respect to those cuts using \vegas\ yields
\result{xs/python/pdf/my_sigma} which is compatible with
\result{xs/python/pdf/sherpa_sigma}, the value \sherpa\ gives.

\subsection{Event Generation and Histograms}%
\label{sec:ppevents}

Generating events of \(\ppgg\) is very similar in principle to
sampling partonic cross section. As before, the range of the \(\eta\)
parameter has to be constrained to obtain physical results. Because
the absolute values of the pseudo rapidities of the two final state
photons are not equal in the lab frame, the shape of the
integration/sampling volume differs from a simple hypercube
\(\Omega\). Furthermore, for the massless limit to be applicable the
center of mass energy of the partonic system must be much greater than
the quark masses. This can be implemented by demanding the transverse
momentum \(p_T\) of a final state photon to be greater than
approximately~\SI{20}{\giga\electronvolt}. A restriction (cut) on
\(p_T\) is suitable because detectors are usually only sensitive above
a certain \(p_T\) threshold and the final state particles have to be
isolated from the beams.

The resulting distribution (without cuts) is depicted in
\cref{fig:dist-pdf} for fixed \(x_2\) and in
\cref{fig:dist-pdf-fixed-eta} for fixed \(\eta\). For \(x_1 = x_2\)
the distribution retains some likeness with the partonic distribution
(see \cref{fig:xs-int-eta}) but gets suppressed for greater values of
\(x_1\). The overall shape of the distribution is clearly highly
sub-optimal for hit-or-miss sampling, only having significant values
when \(x_1\) or \(x_2\) are small (\cref{fig:dist-pdf-fixed-eta}) and
being very steep.

\begin{figure}[ht]
  \centering
  \begin{subfigure}{1\textwidth}
    \centering \plot{pdf/dist3d_x2_const}
    \caption{\label{fig:dist-pdf}Differential cross section convolved
      with PDFs for fixed \protect \result{xs/python/pdf/second_x} in
      picobarn.}
  \end{subfigure}

  \begin{subfigure}{1\textwidth}
    \centering \plot{pdf/dist3d_eta_const}
    \caption{\label{fig:dist-pdf-fixed-eta}Differential cross section
      convolved with PDFs for fixed \protect
      \result{xs/python/pdf/plot_eta} in picobarn.}
  \end{subfigure}
  \caption{\label{fig:dist-pdf-3d}Differential cross section
    convolved with PDFs with one parameter fixed.}
\end{figure}

To remedy that, one has to use a more efficient sampling algorithm
(\vegas) or impose very restrictive cuts. The self-coded
implementation used here can be found in \cref{sec:pycode} and employs
stratified sampling (as discussed in \cref{sec:stratsamp-real}) and
the hit-or-miss method. Because the stratified sampling requires very
accurate upper bounds, they have been overestimated by
\result{xs/python/pdf/overesimate}, which lowers the efficiency
slightly but reduces bias. The monte carlo integrator was used to
estimate the location of the maximum in each hypercube and then this
estimate was improved by a numerical maximize.
% TODO: accuracy of integral in hypercubes

A sample of \result{xs/python/pdf/sample_size} events has been
generated both in \sherpa\ and through own code. The resulting
histograms of some observables are depicted in
\cref{fig:pdf-histos}. The sampling efficiency achieved was
\result{xs/python/pdf/samp_eff} using a total of
\result{xs/python/pdf/num_increments} hypercubes.  As can be seen, the
distributions are compatible with each other. The sherpa runcard
utilized here and the analysis used to produce the histograms can be
found in \cref{sec:ppruncard,sec:ppanalysis}. When comparing
\cref{fig:pdf-eta,fig:histeta} it becomes apparent, that the PDF has
substantial influence on the resulting distribution. Also the center
of momentum energy is not constant anymore and has a steep peak at low
energies due to the steepness of the PDF. The convolution with the pdf
has also smoothed out the jacobian peak seen in \cref{fig:histpt}.

Furthermore new observables have been introduced.  The invariant mass
of the photon pair
\(m_{\gamma\gamma} = (p_{\gamma,1} + p_{\gamma,1})^2\) is the center
of mass energy of the partonic system that produces the photons (see
\cref{eq:ecm_partons}) and proportional to the product of the momentum
fractions of the partons. \Cref{fig:pdf-inv-m} shows, that the vast
majority of the reactions take place at a rather low c.m. energy. Due
to the \(\pt\) cuts the first bin is slightly lower then the second.

The cosines of the scattering angles in the labe frame and the
Collins-Soper (CS) frame are defined in
\cref{eq:sangle,eq:sangle-cs}. The scattering angle is just the angle
between one photon and the photons and the z axis in the c.m. frame if
this frame can be reached by a boost along the z axis\footnote{Or me
  generally, in a z-boosted frame where the angles of the two photons
  are the same.}. Here, the partons are assumed to have no transverse
momentum and therefore the system is symmetric around the beam axis
and therefore this boost is possible. When allowing transverse parton
momenta, as will be done in % TODO: REFERENCE
this symmetry goes away. Defining the z-axis as one beam axis in a
frame would be a quite arbitrary choice that disrespects the symmetry
of the two beams considered here (same energy, identical protons).
Also the random direction of the transverse momentum can add noise
that does not contain much information. The CS frame is defined as the
rest frame of the two outgoing photons in which the z-axis bisects the
angle between the first beam momentum and the inverse momentum of the
second beam. The azimuth angle is measure with respect to a vector
perpendicular to the plane of the beams (which is parallel to the
transverse momentum in the lab frame). In this frame, which was
originally chosen to simplify the extension of the Drell-Yan parton
model to transverse parton momenta~\cite{collins:1977an}, some
symmetry is restored and the study of effects of transverse parton
momenta is facilitated. Because of the above-mentioned symmetry, the
histograms in \cref{fig:pdf-o-angle,fig:pdf-o-angle-cs} are the
same. One would naively expect some likeness to \cref{fig:distcos} but
the cuts imposed alter the distribution quite considerably, cutting of
the \(\cos\theta^\ast\rightarrow 1\)
region. % TODO: come back to that in next chapter

\begin{align}
  \cos\theta^\ast &= \tanh\frac{\eta_1 - \eta_2}{2} \label{eq:sangle}\\
  \cos\theta^*_\text{CS} &= \frac{\sinh(\eta_1 -
                           \eta_2)}{\sqrt{1+(p_{\text{T},1} + p_{\text{T},2})^2/m_{\gamma\gamma}^2}}\cdot
                          \frac{2p_{\text{T},1}p_{\text{T},2}}{m_{\gamma\gamma}^2}\label{eq:sangle-cs}
\end{align}

\begin{figure}[hp]
  \centering
  \begin{subfigure}{.49\textwidth}
    \centering \plot{pdf/eta_hist}
    \caption{\label{fig:pdf-eta} \(\eta\) distribution.}
  \end{subfigure}
  \begin{subfigure}{.49\textwidth}
    \centering \plot{pdf/pt_hist}
    \caption{\label{fig:pdf-pt} \(\pt\) distribution.}
  \end{subfigure}
  \begin{subfigure}{.49\textwidth}
    \centering \plot{pdf/cos_theta_hist}
    \caption{\label{fig:pdf-cos-theta} \(\cos\theta\) distribution.}
  \end{subfigure}
  \begin{subfigure}{.49\textwidth}
    \centering \plot{pdf/inv_m_hist}
    \caption[Histogram of the invariant mass of the final state photon
    system.]{\label{fig:pdf-inv-m} Invariant mass of the
      final state photon system. % This is equal to the center of mass
      % energy of the partonic system before the scattering.
    }
  \end{subfigure}
\end{figure}

\begin{figure}
  \ContinuedFloat
  \begin{subfigure}{.49\textwidth}
    \centering \plot{pdf/o_angle_cs_hist}
    \caption{\label{fig:pdf-o-angle-cs} Scattering angle of the two
      photons in the CS frame.}
  \end{subfigure}
  \begin{subfigure}{.49\textwidth}
    \centering \plot{pdf/o_angle_hist}
    \caption{\label{fig:pdf-o-angle} Scattering angle of the two
      photons in the lab frame.}
  \end{subfigure}
  \caption{\label{fig:pdf-histos}Comparison of histograms of
    observables for \(\ppgg\) generated manually and by \sherpa/\rivet
    and normalized to unity. The sample size was \protect
    \result{xs/python/pdf/sample_size}. }
\end{figure}

%%% Local Variables:
%%% mode: latex
%%% TeX-master: "../../document"
%%% End:
