\chapter{Phenomenological studies of the Diphoton Process Proton-Proton
  Scattering}%
\label{chap:pheno}

In real proton scattering the hard process discussed in
\cref{chap:pdf} is but only a part of the whole picture. Partons do in
general have some intrinsic transverse momentum.  Scattered charges
radiate in both QCD and QED, the former radiation giving rise to
parton-showers and additional transverse momentum of the partons. The
remnants of the proton can radiate showers themselves, scatter in more
or less hard processes (Multiple Interactions, MI) and affect the hard
process through color correlation. All of the processes not directly
connected to the hard process are called the underlying event and have
to be taken into account to generate events that can be compared with
experimental data, as they form a measurable background. Finally the
partons from the showers recombine into hadrons (Hadronization) due to
confinement. This last effect doesn't produce diphoton-relevant
background directly, but affects photon
isolation.~\cite[11]{buckley:2011ge} % TODO: describe isolation

These effects can be calculated or modeled on an per-event base by
modern monte carlo event generators like \sherpa. This has been done
for the diphoton process in a gradual way described in .


%%% Local Variables:
%%% mode: latex
%%% TeX-master: "../document"
%%% End:
