\chapter{Phenomenological Studies of the Diphoton Process in Proton-Proton
  Scattering}%
\label{chap:pheno}

In real proton scattering the hard process discussed in
\cref{chap:pdf} is only a part of the whole picture. Partons do in
general have some intrinsic transverse momentum.  Scattered charges
radiate in both QCD and QED, both giving rise to shower-like cascades
and both can lead to additional transverse momentum of the initial
state partons. The remnants of the proton can radiate showers
themselves, scatter in more or less hard processes (Multiple
Interactions, MI) and affect the hard process through color
correlation. All of the processes not directly connected to the hard
process are called the underlying event and have to be taken into
account to generate events that can be compared with experimental
data. Finally the partons from the showers recombine into hadrons
(hadronization) due to QCD confinement. This last effect doesn't
produce diphoton-relevant background directly, but affects photon
isolation.~\cite[11]{buckley:2011ge}

These effects can be calculated or modeled on an per-event base by
modern Monte Carlo event generators like \sherpa. But these
calculations and models are approximations in most cases. This is done
for the diphoton process in a gradual way described in
\cref{sec:setupan}. Histograms of observables are generated and
discussed in \cref{sec:disco}.

%%% Local Variables:
%%% mode: latex
%%% TeX-master: "../document"
%%% End:
