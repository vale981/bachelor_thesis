\section{Discussion}%
\label{sec:disco}

\begin{table}[ht]
  \centering
  \begin{tabular}{l|SSS}
    &&\multicolumn{2}{c}{events discarded by cuts} \\
    Stage & {\(\sigma\) [\si{\pico\barn}]} & {phase space
                                             [\si{\percent}]} &
                                                                {isolation
                                                                [\SI{1e-4}{\percent}]} \\
    \toprule
    \stfive & 33.02(7) & 97.63 & 9.56 \\
    \stfour & 34.08(7) & 97.56 & 1.89\\
    \stthree & 33.97(7) & 97.56 & 3.52 \\
    \sttwo & 34.60(7) & 97.52 & 3.63 \\
    \stone & 38.74(7) & 96.77 & 0 \\

  \end{tabular}
  \caption{\label{tab:xscut}Cross sections and cut statistics.}
\end{table}

\begin{figure}[ht]
  \centering
  \begin{subfigure}[t]{.49\textwidth}
    \rivethist{pheno/total_pT}
    \caption{\label{fig:disc-total_pT}}
  \end{subfigure}
  \begin{subfigure}[t]{.49\textwidth}
    \rivethist{pheno/azimuthal_angle}
    \caption{\label{fig:disc-azimuthal_angle}}
  \end{subfigure}
  \begin{subfigure}[t]{.49\textwidth}
    \rivethist{pheno/pT}
    \caption{\label{fig:disc-pT}}
  \end{subfigure}
  \begin{subfigure}[t]{.49\textwidth}
    \rivethist{pheno/pT_subl}
    \caption{\label{fig:disc-pT_subl}}
  \end{subfigure}
  \begin{subfigure}[t]{.49\textwidth}
    \rivethist{pheno/inv_m}
    \caption{\label{fig:disc-inv_m}}
  \end{subfigure}
\end{figure}
%
\begin{figure}[t]
  \centering \ContinuedFloat
  \begin{subfigure}[t]{.49\textwidth}
    \rivethist{pheno/cos_theta}
    \caption{\label{fig:disc-cos_theta}}
  \end{subfigure}
  \begin{subfigure}[t]{.49\textwidth}
    \rivethist{pheno/eta}
    \caption{\label{fig:disc-eta}}
  \end{subfigure}
  \begin{subfigure}[t]{.49\textwidth}
    \rivethist{pheno/o_angle}
    \caption{\label{fig:disc-o_angle}}
  \end{subfigure}
  \begin{subfigure}[t]{.49\textwidth}
    \rivethist{pheno/o_angle_cs}
    \caption{\label{fig:disc-o_angle_cs}}
  \end{subfigure}

  \caption{\label{fig:holhistos} Histograms of observables generated
    by simulations with increasingly more effects turned on.}
\end{figure}
%
The results of the \sherpa\ runs for each stage with \(10^7\) events
each are depicted in the histograms in \cref{fig:holhistos} and shall
now be discussed in detail.\footnote{An even higher precision study
  was not possible due to unavailability of access to the
  \emph{Taurus} cluster at the time of writing.}  Because of the
analysis cuts, the total number of accepted events is smaller than the
number of events generated by \sherpa, but sufficient for proper
statistics for most observables. Also the fiducial cross sections of
the different stages, which are compared in \cref{tab:xscut}, differ
as a result of the NLO effects that have been switched on. All
histograms are normalized to their respective cross sections.

When there is no additional \(\pt\), then the photon momenta are back
to back in the plane perpendicular to the beam axis (transverse
plane). Because four-momentum conservation is enforced, every emission
from a parton gives a recoil momentum to that parton. If the system
now gets a recoil from parton showering, then this usually subtracts
\(\pt\) from one of the photons unless that recoil is near
perpendicular to the photons. This leads to an increase in rejected
events. Because the \(\pt\) distribution of the photons
(\cref{fig:disc-pT,fig:disc-pT_subl}) is very steep, a lot of events
produce photons with low \(\pt\) that a prone to falling under the
cuts and so this effect is substantial. The fraction of events that
have been discarded by the phase space cuts are listed in
\cref{tab:xscut} which shows an increase in the fraction of discarded
events for all stages after \stone, contributing to the drop in
fiducial cross section for the \sttwo\ and \stthree stages.

The isolation cuts do affect the observed cross section as well, as is
shown in \cref{tab:xscut}. The fiducial \stfour\ cross section is a
bit higher than the \stthree\ one, because the hadronization favors
isolation of photons by reducing the collinearity of the particles in
the final state and may create particles like neutrinos which are not
detected in the calorimeter or can be easily identified (muons). The
opposite effect can be seen with MI, where the number of final state
particles and thus the hadronic activity in the isolation cone is
increased. This effect leads to another drop in the fiducial cross
section.

The transverse momentum of the photon system (see
\cref{fig:disc-total_pT}) now becomes non trivial, as both the \sttwo\
and \stthree\ stages affect this observable directly. Initial state
radiation generated by the parton showering algorithm kicks the quarks
involved in the hard process and thus generates transverse momentum.
In regions of high \(\pt\) all but the \stone\ stage are largely
compatible, falling off steeply at
\(\mathcal{O}(\SI{10}{\giga\electronvolt})\). Because the parton
shower approximation is only valid in the collinear limit, it can't
necessarily be trusted in higher \(\pt\)
regions~\cite{buckley:2011ge}. The fact that the distribution has a
maximum and falls again towards lower \(\pt\) is related to the nature
of parton shower algorithms, which approximately sum over all terms of
a perturbation series~\cite{buckley:2011ge}.

The partons in a proton are somewhat localized and thus the
uncertainty principle demands that those partons have some momentum
perpendicular to the proton motion. The default parameters in \sherpa\
assign transverse momenta according to a Gaussian distribution with a
mean and standard deviation of \gev{.8}.  In the region of
\SI{1}{\giga\electronvolt} and below, the effects of primordial
\(\pt\) show as an enhancement in the cross section of the \stthree\
stage.
% The distribution for MI is
% enhanced at very low \(\pt\) which could be an isolation effect or
% stem from the fact, that other partons can emit QCD bremsstrahlung and
% showers as well, decreasing the showering probability for the partons
% involved in the hard scattering.

Related effects can be seen in the distribution for the azimuthal
separation of the photons in \cref{fig:disc-azimuthal_angle}. Back to
back photons are favored by all stages because deviations from
this configuration are purely higher order effects, so most events
feature an azimuthal separation of less than \(\pi/2\). The
enhancement of the low \(\pt\) regions in the \stthree\ stage also
leads to an enhancement in the back-to-back region for this stage over
the \sttwo\ stage.

In the \(\pt\) distribution of the leading photon (see
\cref{fig:disc-pT}) the boost of the leading photon towards higher
\(\pt\) in all stages but the \stone\ stage originates from the parton
showering and thus the distribution of those stages are largely
compatible beyond \gev{1}. Again, the effect of primordial \(\pt\)
becomes visible for transverse momenta smaller than \gev{1}.

The \(\pt\) distribution for the sub-leading photon (see
\cref{fig:disc-pT_subl}) shows remarkable resemblance to the \stone\
distribution for all other stages, although there is a very minute
bias to lower \(\pt\). This is consistent with the mechanism described
above so that events that subtract (very small amounts of) \(\pt\)
from the sub-leading second photon are more common. Interestingly, the
effects of primordial \(\pt\) are not very visible.

In leading order, the phase space cuts impose a hard lower bound to
the invariant mass of the photon system. Higher order effects can give
recoil momentum to the partons in such a way, that events with lower
invariant mass pass the cuts. The distribution for the invariant mass
(see \cref{fig:disc-inv_m}) shows that effect. The decline of the
cross section towards lower energies is much steeper than the
PDF-induced decline towards higher energies. High \(\pt\) boost in a
are very rare, which supports the reasoning about the drop in fiducial
cross section. Also, due to the implementation of the showering
algorithm, it may be that only the rather small intrinsic \(\pT\)
changes the center of momentum energy at all.

The angular distributions of the leading photon in
\cref{fig:disc-cos_theta,fig:disc-eta} are most affected by the
differences in total cross section and slightly shifted towards more
central (transverse) regions for all stages from \sttwo\ on due to the
\(\pt\) kicks to the photon system.

Because the diphoton system itself is only affected by recoil to the
initial state quarks, the scattering angle cross sections in
\cref{fig:disc-o_angle,fig:disc-o_angle_cs} show a similar shape in
all stages. Towards small scattering angles, the differences in shape
grow larger, as this is the region where the cuts have the largest
effect. In the CS frame, the cross section does not converge to zero
for \sttwo\ and subsequent stages. With non-zero \(\pt\) of the photon
system, the z-axis of the CS frame rotates out of the region that is
affected by cuts. The ratio plot also shows, that the region where
cross section distributions are similar in shape extends further. In
the CS frame effects of the non-zero \(\pt\) of the photon system are
(somewhat weakly) suppressed.

It becomes clear, that parton showering and the primordial transverse
momentum have the biggest effect on the shape of observables, as they
affect the kinematics of the diphoton process directly. Isolation
effects show most through hadronization and especially multiple
interactions. In observables that exist in leading order
(\(\eta, \pt\), \ldots), the hard process alone gives a reasonably
good qualitative picture, but in most other observables higher order
effects introduce considerable deviations and have to be taken into
account for a realistic study, even with this simple process.

%%% LOCAL Variables:
%%% mode: latex
%%% TeX-master: "../../document"
%%% End:
