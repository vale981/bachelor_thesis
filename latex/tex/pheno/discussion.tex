\section{Discussion}%
\label{sec:disco}
\begin{figure}[ht]
  \centering
  \begin{subfigure}[t]{.49\textwidth}
    \rivethist{pheno/xs}
    \caption{\label{fig:disc-xs}}
  \end{subfigure}
  \begin{subfigure}[t]{.49\textwidth}
    \rivethist{pheno/isolation_discard}
    \caption{\label{fig:disc-iso-disc}}
  \end{subfigure}
  \begin{subfigure}[t]{.49\textwidth}
    \rivethist{pheno/cut_discard}
    \caption{\label{fig:disc-cut-disc}}
  \end{subfigure}
  \caption{Cross section and event discard statistics plots.}
\end{figure}

\begin{figure}[ht]
  \centering
  \begin{subfigure}[t]{.49\textwidth}
    \rivethist{pheno/total_pT}
    \caption{\label{fig:disc-total_pT}}
  \end{subfigure}
  \begin{subfigure}[t]{.49\textwidth}
    \rivethist{pheno/azimuthal_angle}
    \caption{\label{fig:disc-azimuthal_angle}}
  \end{subfigure}
  \begin{subfigure}[t]{.49\textwidth}
    \rivethist{pheno/pT}
    \caption{\label{fig:disc-pT}}
  \end{subfigure}
  \begin{subfigure}[t]{.49\textwidth}
    \rivethist{pheno/pT_subl}
    \caption{\label{fig:disc-pT_subl}}
  \end{subfigure}
  \begin{subfigure}[t]{.49\textwidth}
    \rivethist{pheno/inv_m}
    \caption{\label{fig:disc-inv_m}}
  \end{subfigure}
\end{figure}
%
\begin{figure}[t]
  \centering \ContinuedFloat
  \begin{subfigure}[t]{.49\textwidth}
    \rivethist{pheno/cos_theta}
    \caption{\label{fig:disc-cos_theta}}
  \end{subfigure}
  \begin{subfigure}[t]{.49\textwidth}
    \rivethist{pheno/eta}
    \caption{\label{fig:disc-eta}}
  \end{subfigure}
  \begin{subfigure}[t]{.49\textwidth}
    \rivethist{pheno/o_angle}
    \caption{\label{fig:disc-o_angle}}
  \end{subfigure}
  \begin{subfigure}[t]{.49\textwidth}
    \rivethist{pheno/o_angle_cs}
    \caption{\label{fig:disc-o_angle_cs}}
  \end{subfigure}
  \caption{\label{fig:holhistos} Histograms of observables generated
    by simulations with increasingly more effects turned on.}
\end{figure}
%
The results of the \sherpa\ runs for each stage with \(10^7\) events
each are depicted in the histograms in \cref{fig:holhistos} and shall
now be discussed in detail.\footnote{A higher precision study was not
  possible due to unavailability of access to the \emph{Taurus}
  cluster at the time of writing.}  Because of the analysis cuts, the
total number of accepted events is smaller than the number of events
generated by \sherpa, but sufficient for proper statistics for most
observables. Also the fiducial cross sections of the different stages,
which are compared in \cref{fig:disc-xs}, differ as a result of the
NLO effects that have been switched on. All histograms are normalized
to their respective cross sections.

Effects that give the photon system additional $\pt$ decrease the
cross section This can be understood as follows. When there is no
additional \(\pt\), then the photon momenta are back to back in the
plane perpendicular to the beam axis (transverse plane). Because
four-momentum conservation is enforced, every emission from a parton
gives a recoil momentum to that parton. If the system now gets a
recoil from parton showering, then this usually subtracts \(\pt\) from
one of the photons unless that recoil is near perpendicular to the
photons. Because the \(\pt\) distribution
(\cref{fig:disc-pT,fig:disc-pT_subl}) is very steep, a lot of events
produce photons with low \(\pt\) and so this effect is
substantial. The fraction of events that have been discarded by the
\(\eta\) and \(\pt\) cuts are plotted in \cref{fig:disc-cut-disc},
which shows an increase for all stages after \stone, leading
(principally) to the drop in cross section for the \sttwo\ and
\stthree.

The isolation cuts do affect the cross section as well, as is
demonstrated in \cref{fig:disc-iso-disc} which shows the fraction of
events discarded due to the isolation cuts. The \stfour\ cross section
is a bit higher than the \stthree\ one, because the hardonization
favors isolation of photons by reducing the collinearity of the
particles in the final state and may create particles like neutrinos
that show in the detectors at all or can be easily identified
(muons). The opposite effect can be seen with MI, where the number of
final state particles is increased and this effect leads to another
substantial drop in the cross section.

Also the NLO nature of the effects in the stages after \stone\ reduces
cross section, for instance by ``adding coupling constants'' for each
shower emission or multiple interaction.

The transverse momentum of the photon system (see
\cref{fig:disc-total_pT}) now becomes non trivial, as both the \sttwo\
and \stthree\ stages affect this observable directly. Initial state
radiation generated by the parton showering algorithm kicks the quarks
involved in the hard process and thus generates transverse momentum.
In regions of high \(\pt\) all but the \stone\ stage are largely
compatible, falling off steeply at
\(\mathcal{O}(\SI{10}{\giga\electronvolt})\). Because parton showers
are modeled in the collienar limit, they cannot necessarily be trusted
in higher \(\pt\) regions~\cite{buckley:2011ge}.

The partons in a proton are somewhat localized and thus the
uncertainty principle demands that those partons have some momentum
perpendicular to the proton motion. The default parameters in \sherpa\
assign transverse momenta according to a Gaussian distribution with a
mean and standard deviation of \gev{.8}.  In the region of
\SI{1}{\giga\electronvolt} and below, the effects primordial \(\pt\)
show as an enhancement in the cross section of the \stthree\ stage.
% The distribution for MI is
% enhanced at very low \(\pt\) which could be an isolation effect or
% stem from the fact, that other partons can emit QCD bremsstrahlung and
% showers as well, decreasing the showering probability for the partons
% involved in the hard scattering.
The fact that the distribution has a maximum and falls off towards
lower \(\pt\) relates to the fact, that parton shower algorithms
effectively sum over all terms of the perturbation
series~\cite{buckley:2011ge}.

Related effects can be seen in the distribution for the azimuthal
separation of the photons in \cref{fig:disc-azimuthal_angle}.

Back to back photons are favored by all distributions because
deviations from this configuration are purely NLO effects, so most
events feature an azimuthal separation of less than \(\pi/2\).  The
enhancement of the low \(\pt\) regions in the \stthree\ stage also
leads to an enhancement in the back-to-back region for this stage over
the \sttwo\ stage.

In the \(\pt\) distribution of the leading photon (see
\cref{fig:disc-pT}) the boost of the leading photon towards higher
\(\pt\) in all stages but the \stone\ originates from the parton
showering and thus the distribution of those stages are largely
compatible beyond \gev{1}. Again, the effect of primordial \(\pt\)
becomes visible transverse momenta smaller than \gev{1}.


The \(\pt\) distribution for the sub-leading photon (see
\cref{fig:disc-pT_subl}) shows remarkable resemblance to the \stone\
distribution for all other stages, although there is a very minute
bias to lower \(\pt\). This is consistent with the mechanism described
above so that events that subtract (very small amounts of) \(\pt\)
from the sub-leading second photon are more common. Interestingly, the
effects of primordial \(\pt\) not very visible.

In leading order, the phase space cuts impose a hard lower bound to
the invariant mass of the photon system. Parton showers can give
recoil momentum to the partons in such a way, that events with lower
invariant mass pass the cuts. The distribution for the invariant mass
(see \cref{fig:disc-inv_m}) shows that effect. The decline of the
cross section towards lower energies is much steeper than the
PDF-induced decline towards higher energies. High \(\pt\) boost to
\emph{both} photons are very rare (+ NLO suppressed), which supports
the reasoning about the drop in total cross section.

The angular distributions of the leading photon in
\cref{fig:disc-cos_theta,fig:disc-eta} are most affected by the
differences in total cross section and slightly shifted towards more
central (transverse) regions for all stages from \sttwo\ on due to the
\(\pt\) kicks to the photon system.

Because the diphoton system itself is only affected by recoil to the
initial state quarks, the scattering angle cross sections in
\cref{fig:disc-o_angle,fig:disc-o_angle_cs} show a similar shape in
all stages. Towards small scattering angles, the differences in shape
grow larger, as this is the region where the cuts have the largest
effect. In the CS frame, the cross section does not converge to zero
for \sttwo\ and subsequent stages. With non-zero \(\pt\) of the photon
system, the z-axis of the CS frame rotates out of the region that is
affected by cuts. The ration plot also shows, that the region where
cross section distributions are similar in shape extends further. In
the CS frame effects of the non-zero \(\pt\) of the photon system are
(somewhat weakly) suppressed.

It becomes clear, that parton showering and the primordial transverse
momentum have the biggest effect on the shape of observables, as they
affect the kinematics of the diphoton process directly. Isolation
effects show most through hadronization and especially multiple
interactions. In observables that exist in leading order
(\(\eta, \pt\), \ldots), the hard process alone gives a reasonably
good qualitative picture, but in most other observables non-LO effects
introduce considerable deviations and have to be taken into account
for a realistic study. Even with this simple process.

%%% LOCAL Variables:
%%% mode: latex
%%% TeX-master: "../../document"
%%% End:
