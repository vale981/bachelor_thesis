\section{Set-Up and Analysis}%
\label{sec:setupan}

To observe the impact on the individual aspect of the proton
scattering the following run configurations have been defined. They
are incremental in the sense that each subsequent configuration
extents the previous one.

\begin{description}
\item[Basic] The hard process on parton level as used in \cref{sec:pdf_results}.
\item[PS] The shower generator of \sherpa, \emph{CSS} (dipole-shower),
  is activated and simulates initial state radiation, as there are no
  partons in the final state yet.
\item[PS+pT] The beam remnants and intrinsic parton
  \(\pt\) are simulated, giving rise to final state radiation.
\item[PS+pT+Hadronization] A cluster hadronization model
  implemented in \emph{Ahadic} is activated.
\item[PS+pT+Hadronization+MI] Multiple interactions based on the
  Sj\"ostrand-van-Zijl are simulated.
\end{description}

Detailed description of the implementation of those models can be
found in~\cite{Gleisberg:2008ta} and~\cite{Bothmann:2019yzt}.
%%% Local Variables:
%%% mode: latex
%%% TeX-master: "../../document"
%%% End:
