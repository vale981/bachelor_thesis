\section{Set-Up and Analysis}%
\label{sec:setupan}

To observe the impact on the individual aspect of the proton
scattering the following run configurations have been defined. They
are incremental in the sense that each subsequent configuration
extents the previous one and thus called stages from now on and are
listed below.
%
\begin{description}
\item[LO] The hard process on parton level as used in \cref{sec:pdf_results}.
\item[LO+PS] The shower generator of \sherpa,
  \emph{CSShower}~\cite{schumann2008:ap} (dipole-shower), is activated and
  simulates initial state radiation. The recoil scheme proposed
  in~\cite{hoeche2009:ha}, which has been proven more accurate for
  diphoton production at leading order, has been enabled.
\item[LO+PS+pT] The beam remnants are simulated, giving rise to
  additional radiation and parton showers.  Also the partons are being
  assigned primordial \(\pt\), distributed like a Gaussian with a mean
  value of \SI{.8}{\giga\electronvolt} and a standard deviation of
  \SI{.8}{\giga\electronvolt}\footnote{Those values are \sherpa 's
    defaults.}.
\item[LO+PS+pT+Hadronization] A cluster hadronization model
  implemented in \emph{Ahadic}~\cite{Winter2003:tt} is
  activated. The shower particles are being hadronized and the decay
  of the resulting hadrons simulated if they are unstable.
\item[LO+PS+pT+Hadronization+MI] Multiple Interactions (MI) based on
  the Sj\"ostrand-van-Zijl Model are simulated with
  \emph{Amisic}~\cite{Bothmann:2019yzt}. The MI are parton shower
  corrected, so that there are generally more particles in the final
  state.
\end{description}
%
A detailed description of the implementation of those models can be
found in~\cite{Gleisberg:2008ta} and~\cite{Bothmann:2019yzt}.

The runcard implementing the \texttt{LO+PS+pT+Hadronization+MI}
configuration can be found in \cref{sec:ppruncardfull}. The
\(\abs{\eta}\leq 2.5\) and \(\pt > \SI{20}{\giga\electronvolt}\) cuts
are the same as in \cref{sec:ppxs} and the beam energies have been
chosen as \SI{6500}{\giga\electronvolt} to resemble \lhc\ conditions.
The cuts on the hard process have been loosened to
\(\pt \geq \SI{5}{\giga\electronvolt}\) and \(\abs{\eta}\leq 3\)
because jets and primordial \(\pt\) can increase the final state
\(\pt\) to fall into the analysis cuts.

The analysis is similar to the one used in \cref{sec:ppevents} with
the addition of some observables arising from the possibility of
non-zero transverse momentum of the photon pair:
%
\begin{itemize}
\item total transverse momentum of the photon pair
\item azimuthal angle between the two photons
\item transverse momentum of the sub-leading photon (see below)
\end{itemize}
%
Because the final state now potentially contains additional photons
from hadron decays, the analysis only selects prompt photons with the
highest \(\pt\) (leading photons). Furthermore a cone of
\[R = \sqrt{\qty(\Delta\varphi)^2 + \qty(\Delta\eta)^2} \leq 0.4\]
around each photon must not contain more than \SI{4.5}{\percent} of
the photon transverse momentum (\(+ \SI{6}{\giga\electronvolt}\)), to
simulate experimental photon isolation. The leading photons are
required to have \(\Delta R > 0.45\), to filter photons with
overlapping isolation cones, which would be hard to isolate in
experiments. In truth, the analysis already excludes such photons,
but to be compatible with experimental data, which must rely on such
criteria, they have been included. These cuts are called
\emph{isolation cuts}. The code of the analysis is listed in
\cref{sec:ppanalysisfull} and has been adapted from code privately
communicated by Frank Siegert and Heberth Torres.

The production of photons in showers has not been considered.

%%% Local Variables:
%%% mode: latex
%%% TeX-master: "../../document"
%%% End:
