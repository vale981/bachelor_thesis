\section{Set-Up and Analysis}%
\label{sec:setupan}

To observe the impact on the individual aspect of the proton
scattering the following run configurations have been defined. They
are incremental in the sense that each subsequent configuration
extents the previous one.
%
\begin{description}
\item[LO] The hard process on parton level as used in \cref{sec:pdf_results}.
\item[LO+PS] The shower generator of \sherpa, \emph{CSS} (dipole-shower),
  is activated and simulates initial state radiation, as there are no
  partons in the final state yet.
\item[LO+PS+pT] The beam remnants and intrinsic parton
  \(\pt\) are simulated, giving rise to final state radiation.
\item[LO+PS+pT+Hadronization] A cluster hadronization model
  implemented in \emph{Ahadic} is activated.
\item[LO+PS+pT+Hadronization+MI] Multiple interactions based on the
  Sj\"ostrand-van-Zijl are simulated.
\end{description}
%
A detailed description of the implementation of those models can be
found in~\cite{Gleisberg:2008ta} and~\cite{Bothmann:2019yzt}.

The runcard implementing the \texttt{LO+PS+pT+Hadronization+MI}
configuration can be found in \cref{sec:ppruncardfull}. The
\(\abs{\eta}\leq 2.5\) and \(\pt > \SI{20}{\giga\electronvolt}\) cuts are
the same as in \cref{sec:ppxs} and the beam energies have been chosen
as \SI{6500}{\giga\electronvolt} to resemble \lhc\ conditions.

The analysis is similar to the one used in \cref{sec:ppevents} with
the addition of the observable of total transverse momentum of the
photon pair, which now can be non-zero. Because the final state now
potentially contains additional photons from hadron decays, the
analysis only selects prompt photons with the highest \(\pt\) (leading
photons). Furthermore a cone of
\(R = \sqrt{\qty(\Delta\varphi)^2 + \qty(\Delta\eta)^2} = 0.4\) around
each photon must not contain more than \SI{4.5}{\percent} of the
photon transverse momentum (\(+ \SI{6}{\giga\electronvolt}\)),
attempting to exclude photons stemming from hadron decay are filtered
out. The leading photons are required to have \(\Delta R > 0.45\), to
filter out colinear photons, as they likely stem from hadron
decays. In truth, the analysis already excludes such photons, but to
be compatible with experimental data, which must rely on such
criteria, they have been included. The code of the analysis is listed
in \cref{sec:ppanalysisfull}.

% TODO: refer back to this in discussion
% TODO: cite analysis template

%%% Local Variables:
%%% mode: latex
%%% TeX-master: "../../document"
%%% End:
