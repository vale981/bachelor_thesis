%%% Local Variables: ***
%%% mode:latex ***
%%% TeX-master: "../document.tex"  ***
%%% End: ***

\chapter{The Diphoton Process}%
\label{chap:qqgg}

Consider the scattering of quark and antiquark into two photons
\(\qqgg\), the diphoton process. In leading order this process is
being described by the Feynman diagrams in~\ref{fig:qqggfeyn}.
Because there is only QED involved, the color degrees of freedom
average out and will not be considered henceforth.  Furthermore a high
energy regime will be supposed and therefor masses will be
neglected.

\begin{figure}[h]
  \centering
  \begin{subfigure}[c]{.4\textwidth}
  \centering
  \begin{tikzpicture}
    \begin{feynman}
    \diagram [small,horizontal=i2 to a] {
      i2 [particle=\(q\)] -- [fermion, momentum=\(p_2\)] a --
      [fermion, reversed momentum=\(q\)] b,
      i1 [particle=\(\bar{q}\)] -- [anti fermion, momentum'=\(p_1\)] b,
      i2 -- [opacity=0] i1,
      a -- [photon, momentum=\(p_3\)] f1 [particle=\(\gamma\)],
      b -- [photon, momentum'=\(p_4\)] f2 [particle=\(\gamma\)],
      f1 -- [opacity=0] f2,
    };
    \end{feynman}
  \end{tikzpicture}
  \subcaption{u channel}
  \end{subfigure}
  \begin{subfigure}[c]{.4\textwidth}
  \centering
  \begin{tikzpicture}
    \begin{feynman}
      \diagram [small,horizontal=i2 to a] {
      i2 [particle=\(q\)] -- [fermion, momentum=\(p_2\)] a --
      [fermion, reversed momentum'=\(q\)] b,
      i1 [particle=\(\bar{q}\)] -- [anti fermion, momentum'=\(p_1\)] b,
      i2 -- [opacity=0] i1,
      a -- [draw=none] f2 [particle=\(\gamma\)],
      b -- [draw=none] f1 [particle=\(\gamma\)],
      f1 -- [opacity=0] f2,
    };
    \diagram* {
      (a) -- [photon] (f1),
      (b) -- [photon] (f2),
    };
    \end{feynman}
  \end{tikzpicture}
    \subcaption{\label{fig:qqggfeyn2}t channel}
  \end{subfigure}

  \caption{First order diagrams for \(\qqgg\).}%
  \label{fig:qqggfeyn}
\end{figure}
