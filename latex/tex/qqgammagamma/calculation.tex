\section{Calculation of the Cross Section to Leading Order}%
\label{sec:qqggcalc}

After labeling the incoming quarks and outcoming photons, as well as
the momenta according to \cref{fig:qqggfeyn}, the Feynman rules for
QED yield the matrix elements in \cref{eq:matel}, where \(Z\) is the
electric charge number of the quark and \(g\) is the QED coupling
constant. The respective spinors and polarization vectors are
\(\us,\vs\) and \(\pe\). Parenthesis are being used, whenever indices
would clutter the notation. The matrix element for
\cref{fig:qqggfeyn2} is obtained by simply renaming
\(3\leftrightarrow 4\).
%
\begin{align}
  \label{eq:matel}
  \mathcal{M}_1 &= \frac{(gZ)^2}{\qty(p_1 - p_4)^2}\vsb(1)\pses(4)(\ps_1 -
                \ps_4)\pses(3)\us(2)\\
  \mathcal{M}_2 &= \frac{(gZ)^2}{\qty(p_1 - p_4)^2}\vsb(1)\pses(3)(\ps_1 - \ps_3)\pses(4)\us(2)
\end{align}
%
To simplify notation, some shorthands are intruduced
in \cref{eq:scshort}.
\begin{equation}
  \label{eq:scshort}
  \begin{split}
    s(x) &= \sin(x) & c(x) &= \cos(x) \\ s'(x) &= \sin(\frac{x}{2}) & c'(x) &= \cos(\frac{x}{2})
  \end{split}
\end{equation}

\begin{wrapfigure}{R}{0.4\textwidth}
\centering
\begin{tikzpicture}
  \coordinate (origin) at (0,0);

  \draw[-Latex] (origin) -- (-2,0) node[left] {\(p_3\)};
  \draw[-Latex] (origin) -- (2,0) coordinate (p4) node[right] {\(p_4\)};
  \draw[Latex-,rotate=40] (origin) -- (2,0) coordinate (p2) node[right] {\(p_2\)};
  \draw[Latex-,rotate=40] (origin) -- (-2,0) node[left] {\(p_1\)};
  \draw[fill=black] (origin) circle (.03);

  \draw pic["$\theta$", draw=black, <->, angle eccentricity=1.2, angle radius=1cm] {angle=p4--origin--p2};
\end{tikzpicture}
\caption{\label{fig:qqimpulses} Momentum diagram for the proces
  \(\qqgg\) in the massles limit.}
\end{wrapfigure}
%
All calculations are made with respect to the center of momentum
(c.m.) frame unless stated otherwise. The momenta in the c.m. frame
are concretized in \cref{eq:pchoice} as illustrated in
\cref{fig:qqimpulses}. Note that the photons are aligned to the z-axis
as this led to simpler polarization vectors, when calculating the
matrix element directly. Here casimir's trick will be used but the
labeling was kept.
%
\begin{align}
  \label{eq:pchoice}
  p_1&=p\cdot\mqty(1 \\ s \\ 0 \\ c)
     & p_2&=p\cdot\mqty(1 \\ -s \\ 0 \\ -c)
     & p_3&=p\cdot\mqty(1 \\ 0 \\ 0 \\ -1)
     & p_4&=p\cdot\mqty(1 \\ 0 \\ 0 \\ 1)
\end{align}
%
Now observe that \((p_1-p_4)^2=-4p^2s'^2\) and
\((p_1-p_3)^2=-4p^2c'^2\) (Minkowski metric) and define \(\Gamma_1\)
and \(\Gamma_1\) as in \cref{eq:gammadef}.
%
\begin{align}
  \label{eq:gammadef}
  \Gamma_1 &= \pses(4)(\ps_1 - \ps_4)\pses(3) &
  \Gamma_2 &= \pses(3)(\ps_1 - \ps_3)\pses(4)
\end{align}
%
The total matrix element (the minus sign has been dropped) is given in \cref{eq:totalm}.
\begin{equation}
  \label{eq:totalm}
  \mathcal{M} = \mathcal{M}_1 + \mathcal{M}_2 = \frac{(gZ)^2}{\qty(2p)^2}\vsb(1)\qty(\frac{\Gamma_1}{s'^2}+\frac{\Gamma_2}{c'^2})\us(2)
\end{equation}
%
To obtain an experimentally verifiable cross section the absolute
square of the matrix element has to be averaged over incoming
helicities and summed over all photon polarizations. Using casimir's
trick, the averaging can be simplified to the calculation of a trace
as in \cref{eq:averagedm} where \(s_i\) are helicities, \(\lambda_i\)
are the polarizations and
\(\bar{\Gamma}_i=\gamma^0\Gamma^\dagger_i\gamma^0\). An additional
factor of \(\frac{1}{3}\) arises from the color averaging. There are
total of \emph{nine} color combinations, but only \emph{three}
contribute\footnote{Different color states are orthogonal and there
  are no preferred colors in the beams.}.
%
\begin{equation}
  \label{eq:averagedm}
  \langle\abs{\mathcal{M}}^2\rangle = \frac{1}{4}\sum_{s_1 s_2}\sum_{\lambda_1
    \lambda_2} \abs{\mathcal{M}}^2=\overbrace{\frac{1}{3}\frac{1}{4}\frac{\qty(gZ)^4}{\qty(2p)^4}}^\mathfrak{F}\sum_{\lambda_1
    \lambda_2}\tr[\qty(\frac{\Gamma_1}{s'^2}+\frac{\Gamma_2}{c'^2})
  \ps_2\qty(\frac{\bar{\Gamma}_1}{s'^2}+\frac{\bar{\Gamma}_2}{c'^2})\ps_1]
\end{equation}
%
With the definition \(a_1=4,b_1=3,a_2=3,b_2=4\) the \(\Gamma\)
matrices and their bared variants can be written as in \cref{eq:shortgamma}.
%
\begin{align}
  \label{eq:shortgamma}
  \Gamma_i &= \pses(a_i)(\ps_1 - \ps(a_i))\pses(b_i) & \bar{\Gamma}_i &= \pes(b_i)(\ps_1 - \ps(a_i))\pes(a_i)
\end{align}
%
To work out \cref{eq:averagedm} one must evaluate terms of the
form \cref{eq:gbricks}.
%
\begin{equation}
  \label{eq:gbricks}
  \begin{split}
    \Gamma_{ij} &= \sum_{\lambda_1\lambda_2}
    \tr(\Gamma_i\ps_2\Gamma_j\ps_1)\\
    &= \sum_{\lambda_1\lambda_2}
    \tr[\pses(a_i)(\ps_1-\ps(a_i))\pses(b_i)\ps_2\pes(b_j)(\ps_1 -
    \ps(a_j))\pes(a_j)\ps_1]
  \end{split}
\end{equation}
%
The sum over plarisation can be simplified by utilizing the
completeness relation for polarization vectors for \emph{real} photons
\cref{eq:polcomp}.
%
\begin{equation}
  \label{eq:polcomp}
  \sum_{\lambda=1}^{2}\pe_{(\lambda)}^\mu\pe_{(\lambda)}^{*\nu} = -g^{\mu\nu}
\end{equation}
%
For \(i=j\) \cref{eq:gii} follows by utilizing
\(\gamma_\mu\sl{a}\gamma^\mu=-2\sl{a}\),
\(\gamma_\mu\sl{a}\sl{b}\sl{c}\gamma^\mu=-2\sl{c}\sl{b}\sl{a}\) as
well as \(\ps_i\ps_i=p_i\cdot p_i = 0\) and the well known trace
theorems for the gamma matrices.
\begin{equation}
  \label{eq:gii}
  \begin{split}
\Gamma_{ii} &=
\tr[\gamma_\mu(\ps_1-\ps(a_i))\gamma_\nu\ps_2\gamma^\nu(\ps_1-\ps(a_i))\gamma^\mu\ps_1)]
\\
&= 4\tr[(\ps_1-\ps(a_i))\ps_2(\ps_1-\ps(a_i))\ps_1]\\
&= 32\qty[(p(a_i)\cdot p_2)(p(a_i)\cdot p_1)]
\end{split}
\end{equation}
%
The same tricks as well as the commutation relation for gamma matrices
can be utilized for the case \(i\neq j\) and lead to \cref{eq:gij},
albeit with more technical effort.\footnote{If I learned one thing, it
  is the importance of doing calculations with the utmost verbosity.}
%
\begin{equation}
  \label{eq:gij}
  \begin{split}
\Gamma_{ij} &=
\tr[\gamma_\mu(\ps_1-\ps(a_i))\gamma^\nu\ps_2\gamma^\mu(\ps_1-\ps(a_j))\gamma_\nu\ps_1)]\\
&= -2\tr[\ps_2\gamma^\nu(\ps_1-\ps(a_i))(\ps_1-\ps(a_j))\gamma_\nu\ps_1)]
   \\
&=-16\qty[(p_1p_2)(2(p_ap(a_j)) - p(a_i)p(a_j)) +
(p_1p(a_i))(p_2p(a_j)) - (p_1p(a_j))(p_2p(a_i))]
\end{split}
\end{equation}
%
The crucial step here was to sum over \(\mu\) and utilizing
\(\gamma ^{\mu }\gamma ^{\nu }\gamma ^{\rho }\gamma ^{\sigma }\gamma
_{\mu }=-2\gamma ^{\sigma }\gamma ^{\rho }\gamma ^{\nu }\).


After multiplying out the terms in \cref{eq:averagedm} and plugging in
\cref{eq:gii,eq:gij} there results (with some rather technical
simplifications) the averaged matrix element of
\cref{eq:averagedm_final}. It is noteworthy that the mixing terms
cancel out, in other terms: \(\Gamma_{12} + \Gamma_{21} = 0\). The
result can also be expressed in terms of the pseudo-rapidity
\(\eta \equiv -\ln[\tan(\frac{\theta}{2})]\).
%
\begin{equation}
  \label{eq:averagedm_final}
  \begin{split}
  \langle\abs{\mathcal{M}}^2\rangle &= p^4\cdot\mathfrak{F}\cdot
  32\cdot\qty[\frac{(1-c)(1+c)}{s'^4}] + \qty[\frac{(1-c)(1+c)}{c'^4}] \\
  &= \frac{4}{3}(gZ)^4 \cdot\frac{1+\cos^2(\theta)}{\sin^2(\theta)} =
  \frac{4}{3}(gZ)^4\cdot(2\cosh(\eta) - 1)
  \end{split}
\end{equation}
%
%%% Local Variables: ***
%%% mode:latex ***
%%% TeX-master: "../../document.tex"  ***
%%% End: ***
