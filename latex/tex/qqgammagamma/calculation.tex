\section{Calulation of the Cross Section to first Order}%
\label{sec:qqggcalc}

After labeling the incoming quarks and outcoming photons, as well as
the impulses according to~\ref{fig:qqggfeyn}, the feynman rules yield
the matrix elements in~\eqref{eq:matel}, where \(Q\) is the electric
charge of the quark and \(g\) is the QED coupling constant. The
respective spinors and polarisation vectors are \(\us,\vs\) and
\(\pe\).  The matrix element for~\ref{fig:qqggfeyn2} is obtained by
simply renaming \(3\leftrightarrow 4\). Parenthesis are being used,
whenever indices would clutter the notation.

\begin{align}
  \label{eq:matel}
  \mathcal{M}_1 &= \frac{(gQ)^2}{\qty(p_1 - p_4)^2}\vsb(1)\pses(4)(\ps_1 -
                \ps_4)\pses(3)\us(2)\\
  \mathcal{M}_2 &= \frac{(gQ)^2}{\qty(p_1 - p_4)^2}\vsb(1)\pses(3)(\ps_1 - \ps_3)\pses(4)\us(2)
\end{align}

\begin{wrapfigure}{R}{0.4\textwidth}
\centering
\begin{tikzpicture}
  \coordinate (origin) at (0,0);

  \draw[-Latex] (origin) -- (-2,0) node[left] {\(p_3\)};
  \draw[-Latex] (origin) -- (2,0) coordinate (p4) node[right] {\(p_4\)};
  \draw[Latex-,rotate=40] (origin) -- (2,0) coordinate (p2) node[right] {\(p_2\)};
  \draw[Latex-,rotate=40] (origin) -- (-2,0) node[left] {\(p_1\)};
  \draw[fill=black] (origin) circle (.03);

  \draw pic["$\Theta$", draw=black, <->, angle eccentricity=1.2, angle radius=1cm] {angle=p4--origin--p2};
\end{tikzpicture}
\caption{\label{fig:qqimpulses} Momentum diagram for the proces
  \(\qqgg\) in the massles limit.}
\end{wrapfigure}


To simplify notation, some shorthands are intruduced
in~\eqref{eq:scshort}.
\begin{equation}
  \label{eq:scshort}
  \begin{split}
    s(x) &= \sin(x) & c(x) &= \cos(x) \\ s'(x) &= \sin(\frac{x}{2}) & c'(x) &= \cos(\frac{x}{2})
  \end{split}
\end{equation}

All calculations are made in the center of momentum frame unless
stated otherwise.  The impulses used in the center of momentum frame
are concretised to in~\eqref{eq:pchoice} as well
as~\ref{fig:qqimpulses}.  Note that the photons are aligned to the
z-axis as this led to a simple choice for the polarization vectors,
when calculating the matrix element directly. Here casimir's trick is
being used but the labeling was kept.


\begin{align}
  \label{eq:pchoice}
  p_1&=p\cdot\mqty(1 \\ s \\ 0 \\ c)
     & p_2&=p\cdot\mqty(1 \\ -s \\ 0 \\ -c)
     & p_3&=p\cdot\mqty(1 \\ 0 \\ 0 \\ -1)
     & p_4&=p\cdot\mqty(1 \\ 0 \\ 0 \\ 1)
\end{align}

Now observe that \((p_1-p_4)^2=-4p^2s'^2\) and
\((p_1-p_3)^2=-4p^2c'^2\) (Minkowski metric) and define \(\Gamma_1\)
and \(\Gamma_1\) as in~\eqref{eq:gammadef}.

\begin{align}
  \label{eq:gammadef}
  \Gamma_1 &= \pses(4)(\ps_1 - \ps_4)\pses(3) &
  \Gamma_2 &= \pses(3)(\ps_1 - \ps_3)\pses(4)
\end{align}

The total matrix element (the minus sign has been dropped) is given in~\eqref{eq:totalm}.
\begin{equation}
  \label{eq:totalm}
  \mathcal{M} = \mathcal{M}_1 + \mathcal{M}_2 = \frac{(gQ)^2}{\qty(2p)^2}\vsb(1)\qty(\frac{\Gamma_1}{s'^2}+\frac{\Gamma_2}{c'^2})\us(2)
\end{equation}

To obtain an experimentally verifiable cross section the absolute square of the
matrix element will averaged over incoming helicities and summed over
all photon polarisations.  Using casimir's trick, the averaging can be
simplified to the calculation of a trace as in where \(s_i\) are
helicities, \(\lambda_i\) are the polarisations and \(\bar{\Gamma}_i=\gamma^0\Gamma^\dagger_i\gamma^0\).

\begin{equation}
  \label{eq:averagedm}
  \langle\abs{\mathcal{M}}^2\rangle = \frac{1}{4}\sum_{s_1 s_2}\sum_{\lambda_1
    \lambda_2} \abs{\mathcal{M}}^2=\overbrace{\frac{1}{4}\frac{(gQ)^4}{(2p)^4}}^\mathfrak{F}\sum_{\lambda_1
    \lambda_2}\tr[\qty(\frac{\Gamma_1}{s'^2}+\frac{\Gamma_2}{c'^2})
  \ps_2\qty(\frac{\bar{\Gamma}_1}{s'^2}+\frac{\bar{\Gamma}_2}{c'^2})\ps_1]
\end{equation}

With the definition \(a_1=4,b_1=3,a_2=3,b_2=4\) the \(\Gamma\)
matrices and their bared variants can be written as in~\eqref{eq:shortgamma}.

\begin{align}
  \label{eq:shortgamma}
  \Gamma_i &= \pses(a_i)(\ps_1 - \ps(a_i))\pses(b_i) & \bar{\Gamma}_i &= \pes(b_i)(\ps_1 - \ps(a_i))\pes(a_i)
\end{align}

To work out~\eqref{eq:averagedm} one must evaluate terms of the
form~\eqref{eq:gbricks}.

\begin{equation}
  \label{eq:gbricks}
  \begin{split}
    \Gamma_{ij} &= \sum_{\lambda_1\lambda_2}
    \tr(\Gamma_i\ps_2\Gamma_j\ps_1)\\
    &= \sum_{\lambda_1\lambda_2}
    \tr[\pses(a_i)(\ps_1-\ps(a_i))\pses(b_i)\ps_2\pes(b_i)(\ps_1 -
    \ps(a_i))\pes(a_i)\ps_1]
  \end{split}
\end{equation}

The sum over plarisation can be simplified by utilizing the
completeness relation for polarisation vectors for \emph{external}
photons~\eqref{eq:polcomp}.

\begin{equation}
  \label{eq:polcomp}
  \sum_{\lambda=1}^{2}\pe_{(\lambda)}^\mu\pe_{(\lambda)}^{*\nu} = -g^{\mu\nu}
\end{equation}

For \(i=j\) and by utilizing \(\gamma_\mu\sl{a}\gamma^\mu=-2\sl{a}\),
 \(\gamma_\mu\sl{a}\sl{b}\sl{c}\gamma^\mu=-2\sl{c}\sl{b}\sl{a}\) as
well as \(\ps_i\ps_i=p_i\cdot p_i = 0\) and the well known trace
theorems for the gamma matrices~\eqref{eq:gii} follows.
\begin{equation}
  \label{eq:gii}
  \begin{split}
\Gamma_{ii} &=
\tr[\gamma_\mu(\ps_1-\ps(a_i))\gamma_\nu\ps_2\gamma^\nu(\ps_1-\ps(a_j))\gamma^\mu\ps_1)]
\\
&= 4\tr[(\ps_1-\ps(a_i))\ps_2(\ps_1-\ps(a_i))\ps_1]\\
&= 32\qty[(p(a_i)\cdot p_2)(p(a_i)\cdot p_1)]
\end{split}
\end{equation}

The same tricks as well as the commutation relation for gamma matrices
can be utilized for the case \(i\neq j\) and lead to
\ref{eq:gij}.

\begin{equation}
  \label{eq:gij}
  \begin{split}
\Gamma_{ii} &=
\tr[\gamma_\mu(\ps_1-\ps(a_i))\gamma_\nu\ps_2\gamma^\mu(\ps_1-\ps(a_j))\gamma^\nu\ps_1)]\\
&= 2\tr[\ps_2(\ps_1-\ps(a_i))\gamma_v(\ps_1-\ps(a_j))\gamma^\nu\ps_1]\\
&\hphantom{=} - 2\tr[\gamma_\mu(\ps_1-\ps(a_i))\ps_2(\ps_1-\ps(a_j))\gamma^\mu\ps_1]\\
&\hphantom{=} +
\tr[\gamma_\mu(\ps_1-\ps(a_i))\gamma^\mu\gamma_\nu\ps_2(\ps_1-\ps(a_j))\gamma^\nu\ps_1]\\
&= 32 \qty[2p^2(p(a_j)\cdot p_1) - (p(a_i)\cdot p_2)(p(a_j)\cdot p_1)]
   \\
&\hphantom{=} + 32\qty[(p(a_i)\cdot p(a_j))(p_1\cdot p_2) + (p(a_i)\cdot p_1)(p(a_j)\cdot p_2)]
\end{split}
\end{equation}

The crucial step here was to move the \(\gamma^\mu\) past
\(\gamma_\nu\ps_2\) as shown in~\eqref{eq:movepast}.

\begin{equation}
  \label{eq:movepast}
  \gamma_\nu\ps\gamma^\mu = 2\gamma_\nu p^\mu - 2\delta_\nu^\mu\ps +
  \gamma^\mu\gamma_\nu\ps
\end{equation}

After multiplying out the terms in~\eqref{eq:averagedm} and applying
the \eqref{eq:gii} and~\eqref{eq:gij} there results (after
rather technical simplifications) the
averaged matrix element of~\eqref{eq:averagedm_final}.

\begin{equation}
  \label{eq:averagedm_final}
  \begin{split}
  \langle\abs{\mathcal{M}}^2\rangle &= p^4\cdot\mathfrak{F}\cdot
  32\cdot\qty[\frac{(1-c)(1+c)}{s'^4}] + \qty[\frac{2(1+c) -
    (1+c)(1+c)+4-(1-c)(1-c)}{s'^2c'^2}] \\
  &\hphantom{=} +\qty[\frac{2(1+c) -
    (1+c)(1+c)+4-(1-c)(1-c)}{s'^2c'^2}] +
  \qty[\frac{(1-c)(1+c)}{c'^4}] \\
  &= 4(gQ)^4 \cdot\frac{2+\sin^2(\theta)}{\sin^2(\theta)}
  \end{split}
\end{equation}

The rest is left as an exercise for the reader.
