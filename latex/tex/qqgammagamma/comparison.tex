\section{Discussion and Comparison with \sherpa}%
\label{sec:compsher}

The result obtained in \cref{sec:qqggcalc} shall now be verified by
the MC event generator \sherpa{}~\cite{Gleisberg:2008ta}. To
facilitate this, an expression for the total cross section for a range
of \(\theta\) or \(\eta\) has to be obtained. Using Fermi's golden
rule for \(2\rightarrow 2\) processes and observing that the initial
and final momenta are equal (\(p_i=p_f\)) and \(g=\sqrt{4\pi\alpha}\),
the result \cref{eq:crossec} arises. The differential cross section
has also been calculated in terms of the pseudo-rapidity in
\cref{eq:xs-eta}.

An additional factor of \(\frac{1}{2}\) comes in due to there being
two identical photons in the final state.
\begin{equation}
  \label{eq:crossec}
  \dv{\sigma}{\Omega} =
  \frac{1}{2}\frac{1}{(8\pi)^2}\cdot\frac{\abs{\mathcal{M}}^2}{\ecm^2}\cdot\frac{\abs{p_f}}{\abs{p_i}}
  = \underbrace{\frac{\alpha^2Z^4}{6\ecm^2}}_{\mathfrak{C}}\frac{1+\cos^2(\theta)}{\sin^2(\theta)}
\end{equation}
%
\begin{equation}
  \label{eq:xs-eta}
  \dv{\sigma}{\eta} = 2\pi\cdot\frac{\alpha^2Z^4}{6\ecm^2}\cdot\qty(\tanh(\eta)^2 + 1)
\end{equation}
%
\begin{figure}[ht]
  \centering
  \begin{subfigure}[t]{.49\textwidth}
    \centering \plot{xs/diff_xs_zoom}

    \caption[Plot of the differential cross section of the \(\qqgg\)
    process.]{\label{fig:diffxs_zoom} The differential cross section as a
      function of the polar angle \(\theta\) in the crucial region.}
  \end{subfigure}
  \begin{subfigure}[t]{.49\textwidth}
    \centering \plot{xs/diff_xs}
    \caption[Plot of the differential cross section of the \(\qqgg\)
    process.]{\label{fig:diffxs} The differential cross section as a
      function of the polar angle \(\theta\) over the full integration
      interval. }
  \end{subfigure}
  \begin{subfigure}[t]{.49\textwidth}
  \centering
  \plot{xs/total_xs}
  \caption[Plot of the total cross section of the \(\qqgg\)
  process.]{\label{fig:totxs} The cross section
    \cref{eq:total-crossec} of the process for a pseudo-rapidity
    integrated over \([-\eta, \eta]\).}
\end{subfigure}
\caption{\label{fig:xsfirst} Plots of the differential and total cross section
  for \(\qqgg\).}
\end{figure}
%
The differential cross section \cref{eq:crossec} (see also
\cref{fig:diffxs}) is divergent for angles near zero or \(\pi\), as
expected for massless t- or u-channel diagrams, but remains finite in
the physical region (see \cref{fig:diffxs_zoom}). At small scattering
angles the absolute square of the momentum carried by the virtual
quark in \cref{fig:qqggfeyn} goes to zero, which in the mass-less
limit means that the virtual quark comes \emph{on-shell} and thus the
process becomes resonant. The divergence of the cross section itself
is also not the problem here, because it can be transformed into a
form where the divergence does not occur (see \cref{eq:xs-eta}).

The differential cross section clearly is symmetric around
\(\theta=\frac{\pi}{2}\) as was to be expected, because the photons
are indistinguishable. To compare the cross section to experiment and
to simulation an interval around \(\theta=\frac{\pi}{2}\) has to be
chosen, where the leading order, mass-less approximation may yield a
physical result.

The total cross section in such an interval, given by
integrating \cref{eq:crossec} for \(\theta\in [\theta_1, \theta_2]\)
or \(\eta\in [\eta_1, \eta_2]\) is given
in \cref{eq:total-crossec}.
%
\begin{equation}
  \label{eq:total-crossec}
  \begin{split}
  \sigma &=
  2\pi\mathfrak{C}\cdot\qty{\cos(\theta_2)-\cos(\theta_1)+2\qty[\artanh(\cos(\theta_1))
    - \artanh(\cos(\theta_2))]} \\
  &=2\pi\mathfrak{C}\cdot\qty[\tanh(\eta_2) - \tanh(\eta_1) + 2(\eta_1
  - \eta_2))] \\
  &={\frac{\pi\alpha^2Z^4}{3\ecm^2}}\cdot\qty[\tanh(\eta_2) - \tanh(\eta_1) + 2(\eta_1
  - \eta_2))]
  \end{split}
\end{equation}
%
As can be seen in \cref{fig:totxs}, the cross section, integrated over
an interval of \([-\eta, \eta]\), is dominated by the linear
contributions in \cref{eq:total-crossec} and would result in an
infinity if no cut on \(\eta\) would be imposed. Choosing
\result{xs/python/eta} and \result{xs/python/ecm} the
process was MC integrated in \sherpa\ using the runcard in
\cref{sec:qqggruncard}. This runcard describes the exact same (leading
order) process as the calculated cross section.

\sherpa\ arrives at the cross section
\result{xs/python/xs_sherpa}. Plugging the same parameters into
\cref{eq:total-crossec} gives \result{xs/python/xs} which is within
the uncertainty range of the \sherpa\ value. This verifies the result
for the total cross section.

%%% Local Variables:
%%% mode: latex
%%% TeX-master: "../../document"
%%% End:
