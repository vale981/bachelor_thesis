\section{Discussion and Comparison with Sherpa}%
\label{sec:compsher}

The result obtained in~\ref{sec:qqggcalc} shall now be verified by
monte-carlo in \verb|Sherpa|. To facilitate this, an expression for
the total cross section for a range of \(\theta\) or \(\eta\) has to
be obtained. Using the golden rule for \(2\rightarrow 2\) processes
and observing that the initial and final impulses are equal
(\(p_i=p_f\)) and \(g=\sqrt{4\pi\alpha}\), the
result~\eqref{eq:crossec} arises.

An additional factor of \(\frac{1}{2}\) comes in due to there being
two identical photons in the final state.
\begin{equation}
  \label{eq:crossec}
  \dv{\sigma}{\Omega} =
  \frac{1}{2}\frac{1}{(8\pi)^2}\cdot\frac{\abs{\mathcal{M}}^2}{\ecm^2}\cdot\frac{\abs{p_f}}{\abs{p_i}}
  = \overbrace{\frac{\alpha^2Q^4}{6\ecm^2}}^{\mathfrak{C}}\frac{1+\cos^2(\theta)}{\sin^2(\theta)}
\end{equation}


\begin{figure}[ht]
  \centering
  \plot{xs/diff_xs}
  \caption[Plot of the differential cross section of the \(\qqgg\)
  process.]{\label{fig:diffxs} The differential cross section of the
    process \(\qqgg\) as a function of the azimuth angle
    \(\theta\). The pseudo-rapdity cut \(\abs{\eta} \leq 2.5\) is
    being visualized.}
\end{figure}

The differential cross section~\eqref{eq:crossec} (see
also~\ref{fig:diffxs}) is divergent for angles near zero or
\(\pi\). Allowing finite mass in the calculation may regularize this
divergence. Because \(m=0\) is the limit for
\(\ecm\rightarrow\infty\), the cross section would still have strong
peaks for angles near \(0,\pi\) at high energies so that the result is
not altogether nonphysical. It is clearly symmetric around
\(\theta=\frac{pi}{2}\) as was to be expected, because the photons are
indistinguishable. To compare the cross section to experiment and to
simulation an interval around \(\theta=\frac{\pi}{2}\) has to be
chosen, where the first order, mass-less approximation may yield
sensible results.

The total cross section in such an interval, given by
integrating~\eqref{eq:crossec} for \(\theta\in [\theta_1, \theta_2]\)
or \(\eta\in [\eta_1, \eta_2]\) is given in~\eqref{eq:total-crossec}.

\begin{equation}
  \label{eq:total-crossec}
  \begin{split}
  \sigma &=
  2\pi\mathfrak{C}\cdot\qty{\cos(\theta_2)-\cos(\theta_1)+2\qty[\artanh(\cos(\theta_1))
    - \artanh(\cos(\theta_2))]} \\
  &=2\pi\mathfrak{C}\cdot\qty[\tanh(\eta_1) - \tanh(\eta_2) + 2(\eta_2
  - \eta_1))] \\
  &={\frac{\pi\alpha^2Q^4}{3\ecm^2}}\cdot\qty[\tanh(\eta_1) - \tanh(\eta_2) + 2(\eta_2
  - \eta_1))]
  \end{split}
\end{equation}

Choosing \(\eta\in [-2.5,2.5]\) and
\(\ecm=\SI{100}{\giga\electronvolt}\) the process was monte carlo
integrated in sherpa using the runcard in~\ref{sec:qqggruncard}. This
runcard describes the exact same (first order) process as the
calculated cross section.

Sherpa yields \(\sigma = \SI{0.05380\pm
  .00005}{\pico\barn}\). Plugging the same parameters
into~\eqref{eq:total-crossec} gives ../../prog/python/qqgg/results/xs.tex which is
within the uncertainty range of the Sherpa value.
