\chapter{Summary and Outlook}%
\label{chap:summary}

In this thesis, the leading order analytical cross section for the
quark level diphoton process has been calculated and verified using
the \sherpa\ event generator. Subsequently some Monte Carlo methods
for integration and sampling were mathematically motivated,
implemented and applied to the diphoton process, resulting in the
implementation of a simple event generator for proton-proton
scattering. Good sampling efficiency was achieved, at the cost of
accuracy.

Finally a phenomenological study of the diphoton process in
proton-proton scattering was performed by incrementally enabling
additional effects in the \sherpa\ event generator. Even with the
simplistic leading order matrix element, NLO effects like parton
showering showed significant impact on certain observables.

The simplistic implementation of the diphoton process could be
developed further by using NLO matrix elements for the hard
process. This would lead extra emissions and new requirements for
photon isolation and a plethora of new effects. Furthermore the impact
hard photons from parton showers should can be studied.

%%% LOCAL Variables:
%%% mode: latex
%%% TeX-master: "../document"
%%% End:
