\chapter{Summary and Outlook}%
\label{chap:summary}

In this thesis, the leading order analytical cross section for the
quark level diphoton process has been calculated and verified using
the \sherpa\ event generator. Subsequently some Monte Carlo methods
for integration and sampling were mathematically motivated,
implemented and applied to the diphoton process, resulting in the
implementation of a simple event generator for proton-proton
scattering. Good sampling efficiency was achieved and potential
problems with the employed algorithm were highlighted.

Finally a phenomenological study of the diphoton process in
proton-proton scattering was performed by incrementally enabling
additional effects in the \sherpa\ event generator. Albeit the leading
order matrix element gives a good qualitative picture for the shape of
some observables, higher order effects like parton showering proved to
have a significant impact on certain observables.

The simplistic implementation of the diphoton process could be
developed further by using NLO matrix elements for the hard
process. This would lead to extra emissions and new requirements for
photon isolation and a plethora of new effects. Another avenue of
refinement of the simulation would be to allow the creation of photons
in parton showers. The impact of increased photon activity could lead
to additional observable effects.


%%% LOCAL Variables:
%%% mode: latex
%%% TeX-master: "../document"
%%% End:
