\documentclass[10pt, aspectratio=169]{beamer}
\usepackage{siunitx}
\usepackage[T1]{fontenc}
\usepackage{booktabs}
\usepackage[backend=biber]{biblatex}
\usepackage{mathtools,amssymb}
\usepackage{physics}
\usepackage{slashed}
\usepackage{tikz}
\usepackage{tikz-feynman}
\usepackage[list=true, font=small,
labelformat=brace, position=top]{subcaption}
% \setbeameroption{show notes on second screen} %
\addbibresource{thesis.bib}
\graphicspath{ {figs/} }

\usetheme{Antibes}
\usepackage{eulerpx}
\usepackage{ifdraft}

\usefonttheme[onlymath]{serif}
\setbeamertemplate{itemize items}[default]
\setbeamertemplate{enumerate items}[default]
\AtBeginSection[]
{
   \begin{frame}
       \tableofcontents[currentsection]
   \end{frame}
}

\setbeamertemplate{footline}[frame number]
\setbeamertemplate{note page}[plain]

\sisetup{separate-uncertainty = true}
% Macros

%% qqgg
\newcommand{\qqgg}[0]{q\bar{q}\rightarrow\gamma\gamma}

%% ppgg
\newcommand{\ppgg}[0]{pp\rightarrow\gamma\gamma}

%% Momenta and Polarization Vectors convenience
\DeclareMathOperator{\ps}{\slashed{p}}

\DeclareMathOperator{\pe}{\varepsilon}
\DeclareMathOperator{\pes}{\slashed{\pe}}

\DeclareMathOperator{\pse}{\varepsilon^{*}}
\DeclareMathOperator{\pses}{\slashed{\pe}^{*}}

%% Spinor convenience
\DeclareMathOperator{\us}{u}
\DeclareMathOperator{\usb}{\bar{u}}

\DeclareMathOperator{\vs}{v}
\DeclareMathOperator*{\vsb}{\overline{v}}

%% Center of Mass energy
\DeclareMathOperator{\ecm}{E_{\text{CM}}}

%% area hyperbolicus
\DeclareMathOperator{\artanh}{artanh}
\DeclareMathOperator{\arcosh}{arcosh}

%% Fast Slash
\let\sl\slashed

%% Notes on Equations
\newcommand{\shorteqnote}[1]{ &  & \text{\small\llap{#1}}}

%% Typewriter Macros
\newcommand{\sherpa}{\texttt{Sherpa}}
\newcommand{\rivet}{\texttt{Rivet}}
\newcommand{\vegas}{\texttt{VEGAS}}
\newcommand{\lhapdf}{\texttt{LHAPDF6}}
\newcommand{\scipy}{\texttt{scipy}}

%% Sherpa Versions
\newcommand{\oldsherpa}{\texttt{2.2.10}}
\newcommand{\newsherpa}{\texttt{3.0.0} (unreleased)}

%% Special Names
\newcommand{\lhc}{\emph{LHC}}

%% Expected Value and Variance
\newcommand{\EX}[1]{\operatorname{E}\qty[#1]}
\newcommand{\VAR}[1]{\operatorname{VAR}\qty[#1]}

%% Uppercase Rho
\newcommand{\Rho}{P}

%% Transverse Momentum
\newcommand{\pt}[0]{p_\mathrm{T}}

%% Sign Function
\DeclareMathOperator{\sign}{sgn}

%% Stages
\newcommand{\stone}{\texttt{LO}}
\newcommand{\sttwo}{\texttt{LO+PS}}
\newcommand{\stthree}{\texttt{LO+PS+pT}}
\newcommand{\stfour}{\texttt{LO+PS+pT+Hadr.}}
\newcommand{\stfive}{\texttt{LO+PS+pT+Hadr.+MI}}

%% GeV
\newcommand{\gev}[1]{\SI{#1}{\giga\electronvolt}}

%% Including plots
\newcommand{\plot}[2][,]{%
  \includegraphics[draft=false,#1]{./figs/#2.pdf}}
\newcommand{\rivethist}[2][,]{%
  \includegraphics[draft=false,width=\textwidth,#1]{./figs/rivet/#2.pdf}}

%% Including Results
\newcommand{\result}[1]{\input{./results/#1}\!}

\title{A Study of Monte Carlo Methods and their Application to
  Diphoton Production at the Large Hadron Collider}
\subtitle{Bachelorvortrag}
\author{Valentin Boettcher}
\beamertemplatenavigationsymbolsempty

\begin{document}
\hypersetup{pageanchor=false}
\maketitle

\hypersetup{pageanchor=true}
\pagenumbering{arabic}

\begin{frame}
  \tableofcontents
\end{frame}

\section{Introduction}
\begin{frame}{Motivation}
  \begin{block}{Monte Carlo Methods}
    \begin{itemize}
    \item (most) important numerical tools (not just) in particle
      physics
    \item crucial interface of theory and experiment
    \item enable precision predictions within and beyond SM
    \end{itemize}
  \end{block}
  \pause
  \begin{block}{Diphoton Process \(\qqgg\)}
    \begin{itemize}
    \item simple QED process, calculable by hand
    \item higgs decay channel: \(H\rightarrow \gamma\gamma\)
      \begin{itemize}
      \item instrumental in its
        discovery~\cite{Aad:2012tfa,Chatrchyan:2012ufa}
      \end{itemize}
    \item dihiggs decay \(HH\rightarrow b\bar{b}\gamma\gamma\)
      \begin{itemize}
      \item process of recent interest~\cite{aaboud2018:sf}
      \end{itemize}
    \end{itemize}
  \end{block}
\end{frame}

\section{Calculation of the \(\qqgg\) Cross Section}
\subsection{Approach}
\begin{frame}
  \begin{columns}[T]
    \begin{column}{.5\textwidth}
      \begin{figure}[ht]
        \centering
        \begin{subfigure}[c]{.28\textwidth}
          \centering
          \begin{tikzpicture}[scale=.6]
            \begin{feynman}
              \diagram [small,horizontal=i2 to a] { i2
                [particle=\(q\)] -- [fermion, momentum=\(p_2\)] a --
                [fermion, reversed momentum=\(q\)] b, i1
                [particle=\(\bar{q}\)] -- [anti fermion,
                momentum'=\(p_1\)] b, i2 -- [opacity=0] i1, a --
                [photon, momentum=\(p_3\)] f1 [particle=\(\gamma\)], b
                -- [photon, momentum'=\(p_4\)] f2
                [particle=\(\gamma\)], f1 -- [opacity=0] f2, };
            \end{feynman}
          \end{tikzpicture}
          \subcaption{u channel}
        \end{subfigure}
        \begin{subfigure}[c]{.28\textwidth}
          \centering
          \begin{tikzpicture}[scale=.6]
            \begin{feynman}
              \diagram [small,horizontal=i2 to a] { i2
                [particle=\(q\)] -- [fermion, momentum=\(p_2\)] a --
                [fermion, reversed momentum'=\(q\)] b, i1
                [particle=\(\bar{q}\)] -- [anti fermion,
                momentum'=\(p_1\)] b, i2 -- [opacity=0] i1, a --
                [draw=none] f2 [particle=\(\gamma\)], b -- [draw=none]
                f1 [particle=\(\gamma\)], f1 -- [opacity=0] f2, };
              \diagram* { (a) -- [photon] (f1), (b) -- [photon] (f2),
              };
            \end{feynman}
          \end{tikzpicture}
          \subcaption{\label{fig:qqggfeyn2}t channel}
        \end{subfigure}
%
        \caption{Leading order diagrams for \(\qqgg\).}%
        \label{fig:qqggfeyn}
      \end{figure}
    \end{column}
    \pause
    \begin{column}{.5\textwidth}
      \begin{block}{Task: calculate
          \(\abs{\mathcal{M}}^2\)}
        \begin{enumerate}[<+->]
        \item translate diagrams to
          matrix elements
        \item use Casimir's trick to
          average over spins
        \item use completeness
          relation to sum over
          photon polarizations
        \item use trace identities
          to compute the absolute
          square
        \item simplify with
          trigonometric identities
        \end{enumerate}
      \end{block}
      \pause Here: Quark masses
      neglected.
    \end{column}
  \end{columns}
\end{frame}

\subsection{Result}

\begin{frame}
  \begin{equation}
    \label{eq:averagedm_final}
    \langle\abs{\mathcal{M}}^2\rangle = \frac{4}{3}(gZ)^4
    \cdot\frac{1+\cos^2(\theta)}{\sin^2(\theta)} =
    \frac{4}{3}(gZ)^4\cdot(2\cosh(\eta) - 1)
  \end{equation}
  %
  \pause
  \[\overset{\text{Golden Rule}}{\implies}\]
  \pause
  \begin{equation}
    \label{eq:crossec}
    \dv{\sigma}{\Omega} =
    \frac{1}{2}\frac{1}{(8\pi)^2}\cdot\frac{\abs{\mathcal{M}}^2}{\ecm^2}\cdot\frac{\abs{p_f}}{\abs{p_i}}
    = \underbrace{\frac{\alpha^2Z^4}{6\ecm^2}}_{\mathfrak{C}}\frac{1+\cos^2(\theta)}{\sin^2(\theta)}
  \end{equation}

  \pause
  \begin{figure}[ht]
    \centering
    \begin{minipage}[c]{0.3\textwidth}
      \plot[scale=.6]{xs/diff_xs}
    \end{minipage}
    \begin{minipage}[c]{0.3\textwidth}
      \caption{The differential cross section as a function of the
        polar angle \(\theta\).}
    \end{minipage}
  \end{figure}
\end{frame}

\begin{frame}{Comparison with \sherpa}
  \begin{itemize}
  \item<1-> choose \result{xs/python/eta} and \result{xs/python/ecm} and
    integrate XS
    \begin{equation}
      \label{eq:total-crossec}
      \sigma = {\frac{\pi\alpha^2Z^4}{3\ecm^2}}\cdot\qty[\tanh(\eta_2) - \tanh(\eta_1) + 2(\eta_1
      - \eta_2)]
    \end{equation}
  \item<2-> analytical result: \result{xs/python/xs}
  \item<3-> compatible with \sherpa: \result{xs/python/xs_sherpa}
  \end{itemize}
  \begin{figure}[ht]
    \centering
    \begin{minipage}[c]{0.3\textwidth}
      \plot[scale=.5]{xs/total_xs}
    \end{minipage}
    \begin{minipage}[c]{0.3\textwidth}
      \caption{\label{fig:totxs} The cross section
        of the process for a pseudo-rapidity
        integrated over \([-\eta, \eta]\).}
    \end{minipage}
  \end{figure}
\end{frame}
\end{document}
massless limit
